% !TEX TS-program = pdflatexmk
\documentclass[12pt]{article}

% Layout.
\usepackage[top=1in, bottom=0.75in, left=1in, right=1in, headheight=1in, headsep=6pt]{geometry}

% Fonts.
\usepackage{mathptmx}
\usepackage[scaled=0.86]{helvet}
\renewcommand{\emph}[1]{\textsf{\textbf{#1}}}
\renewcommand{\familydefault}{\sfdefault}

% Misc packages.
\usepackage{amsmath,amssymb,latexsym}
\usepackage{graphicx,hyperref}
\usepackage{array}
\usepackage{xcolor}
\usepackage{multicol}
\usepackage{tabularx,colortbl}
\usepackage{enumitem}

\usepackage{fancyhdr}
\pagestyle{fancy} 
\lhead{\large\sf\textbf{MATH 316: History of Math}}
\rhead{\large\sf\textbf{Project Ideas}}

\begin{document}
\begin{itemize}
\item Apollonius' \textit{Conics}
\item Nicole Oresme's \textit{The Latitude of Forms} where he begins to thing about how rates of change have to do with distance travelled.
\item Srinivasa Ramanujan's Squaring the Circle or other works. (One must choose carefully. UAF has a book called: The Collected Papers of Srinivasa Ramanujan.)
\item Francois Viete, \textit{The New Algebra}
\item Simon Stevin and Decimal Fractions
\item John Napier and Logarithms
\item Leonhard Euler and the Problem of the Seven Bridges of Konigsberg or Theorems on Residues Obtained by the Division of Powers
\item Archimedes Sphere and Cylinder Propositions 33 and 34
\item Ptolemy and the beginning of the sine function (On the lengths of chords in a circle)
\item John Wallis on Imaginary Numbers (and thinking about negative numbers, square roots of negative numbers, and beginning graphing)
\item Pascal on the Arithmetic Triangle
\item perfect and amicable numbers
\item the slide rule
\end{itemize}
\end{document}