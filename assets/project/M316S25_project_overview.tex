% !TEX TS-program = pdflatexmk
\documentclass[12pt]{article}

% Layout.
\usepackage[top=1in, bottom=0.75in, left=1in, right=1in, headheight=1in, headsep=6pt]{geometry}

% Fonts.
\usepackage{mathptmx}
\usepackage[scaled=0.86]{helvet}
\renewcommand{\emph}[1]{\textsf{\textbf{#1}}}
\renewcommand{\familydefault}{\sfdefault}

% Misc packages.
\usepackage{amsmath,amssymb,latexsym}
\usepackage{graphicx,hyperref}
\usepackage{array}
\usepackage{xcolor}
\usepackage{multicol}
\usepackage{tabularx,colortbl}
\usepackage{enumitem}

\usepackage{fancyhdr}
\pagestyle{fancy} 
\lhead{\large\sf\textbf{MATH 316: History of Math}}
\rhead{\large\sf\textbf{Project Overview}}

\begin{document}
{\large{The Project}}

A complete project consists of a written paper, a deck of slides, and an oral presentation to the class. 

The paper does not need to be long but it does need to meet the expectations (see below). In particular, the paper must include an \emph{explanation of historical mathematics}, how it differs from our modern view, and where this particular work fits into the larger mathematical history.

Each student will also give a 15-20 minute presentation to the class about their topic. Since this is a hybrid course, all presentation will need to be delivered online using slides (i.e. no whiteboard talks).

The mathematical topic of the project must be approved in advance and must be something an undergraduate who has just completed Calculus III can understand.\\

{\large{Goals}}

\begin{itemize}
\item Give each student the opportunity to learn more about an area of the history of mathematics interesting to them.
\item Give all students in the class the benefit of learning more about this topic.
\item Experience reading some mathematics in its original form.
\item Practice writing and explaining math to others. 
\end{itemize}

{\large{Expectations}}

A complete project will include all of the items below.
\begin{itemize}
\item An explanation of a piece of historical mathematics (e.g. the Babylonian's solution to a quadratic equation, Diophantus' articulation and solution to a problem in \textit{Arithmetica}, Euclid's proof the the Pythagorean Theorem from \textit{Elements}).
\item Some background biographical information on the author.
\item A discussion of how the author's approach is different from and/or similar to our modern understanding.
\item A discussion of how this idea/theorem/treatise/proof fits into the broader view of the history of mathematics.
\item An ideal ($A$-quality) paper will also include reading and snippets from an original historical source (or sources).
\end{itemize}

{\large{A Place to Start}}

There are two natural places to start: by topic or by document.

Starting by topic means doing a little research on the origins of a topic you are interested in. (What are the origins of Graph Theory?) or a topic we glossed over in class (What about those perfect and amicable numbers?.)

Starting by document means looking through a variety original documents easily available to you and picking something that is interesting. (See links on webpage).



\end{document}