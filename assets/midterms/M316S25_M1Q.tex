% !TEX TS-program = pdflatexmk
\documentclass[12pt]{article}

% Layout.
\usepackage[top=1in, bottom=0.75in, left=1in, right=1in, headheight=1in, headsep=6pt]{geometry}

% Fonts.
\usepackage{mathptmx}
\usepackage[scaled=0.86]{helvet}
\renewcommand{\emph}[1]{\textsf{\textbf{#1}}}

% Misc packages.
\usepackage{amsmath,amssymb,latexsym}
\usepackage{graphicx,hyperref}
\usepackage{array}
\usepackage{xcolor}
\usepackage{multicol}
\usepackage{tabularx,colortbl}
\usepackage{enumitem}

\usepackage{fancyhdr}
\pagestyle{fancy} 
\lhead{\large\sf\textbf{MATH 316: History of Math}}
\rhead{\large\sf\textbf{Midterm I Questions)}}

\begin{document}
\begin{enumerate}
\item This question is about Egyptian arithmetic and numerical representation.
	\begin{enumerate}
	\item Multiply 173 by 12 and divide 173 by 12 using the ancient Egyptian method of doubling. Show your work.
	\item Demonstrate the the decomposition of a fraction into unique unit fractions is not unique.
	\item Explain the purpose of the table that begins the Rhind Papyrus as described in your text book: ``At the beginning of the Rhind Papyrus, there is ... a table giving the breakdown for fractions with numerator 2 and denominator an odd number between 5 and 101."
	\end{enumerate}
\item For this problem we will use the text's notation for numerals base 60.
	\begin{enumerate}
	\item Explain, with computation, how $3;20$ can be the reciprocal of 18.
	\item What would be the analog of the reciprocal of 18 in base 10.
	\item Describe two advantages of the Babylonian base 60 representation of numbers.
	\end{enumerate}
\item Know about the method of false position. You should know how to solve a problem using this method, when the method is appropriate and what its advantages are over our modern methods.
\item Give two concrete examples from our textbook in support of the statement that \emph{how we \underline{write} mathematics impacts how we \underline{do} mathematics}. Each example requires the comparison of two different cultures, their writing, and their mathematics. Full points for breadth of your examples. That is, avoid having both examples be about numerical representation.
\item This question is about our knowledge of ancient Egyptian mathematics from the Rhind and Moscow papyri.
	\begin{enumerate}
	\item Provide some context: when, where, and what are these documents made of
	\item What sort of mathematics is in these texts and what is the nature of the exposition.
	\item Describe three specific results.
	\end{enumerate}
\item Give a concrete example of cut-and-paste algebra. This requires describing a particular text, time period, algebraic problem, and cut-and-paste picture solution.
\item This question is about the curve called the \textbf{quadtratrix}.
	\begin{enumerate}
	\item How is it defined, by whom, roughly when.
	\item Why is it important in the history of mathematics? Give two specific reasons.
	\end{enumerate}
\item Give a brief argument in favor of and in opposition to the statement that ancient Babylonians knew the quadratic equation and/or the Pythagorean Theorem.
\item Incommensurable Quantities
	\begin{enumerate}
	\item What did ancient Greek mathematicians mean by the words ``line segments AB and BC are incommensurable"?
	\item How, when and who discovered the existence of incommensurable line segments. 
	\item Describe (with pictures) how ancient Greek mathematicians could have constructed a line segment of length $\sqrt{n}$ for any positive integer $n.$
	\item Why is their existence a crisis for mathematicians?
	\end{enumerate}
\item What are the three construction problems of antiquity and why are they important?
\item The following question is about the quadrature of regions.
	\begin{enumerate}
	\item Demonstrate how to construct the quadrature of an arbitrary rectangle or triangle.
	\item What is a lune and what is meant by Hippocrates' quadrature of a lune.
	\item Why is the quadrature of a lune important?
	\end{enumerate}
\item This question is about Euclid's Elements.
	\begin{enumerate}
	\item Provide some basic historical facts: when and where was it written
	\item Describe its contents. What was its structure? What mathematical topics? Its style of exposition?
	\item What were some specific notable results in the Elements?
	\item State the 5th Postulate and describe its importance to the Elements and to the history of mathematics.
	\item Explain what is meant by geometric algebra using the solutions to $ac=bc$ as an example.
	\item How does Euclid's proof of the infinitude of primes differ from a modern one?
	\item Describe its influence on mathematics.
	\end{enumerate}
\item This question is about some of the mathematics of Archimedes.
	\begin{enumerate}
	\item Provide some basic historical facts: when and where did Archimedes live.
	\item Give a brief comparison between Archimedes and Euclid (time, location, nature of writing)
	\item Describe the method Archimedes used to estimate the circumference of a circle.
	\item Explain how estimating that the circumference a circle of radius 20 is about 63.1 can be interpreted as an estimation of $\pi$.
	\item Describe the method Archimedes used in his quadrature of a parabolic segment.
	\end{enumerate}
\item This question is about Diophantus' work in \textit{Arithmetica}.
	\begin{enumerate}
	\item When this text was written and what sort of problems does it contain?
	\item Be able to solve a problem from \textit{Arithmetica}  that you have seen before but with slightly different numbers. 
	\item Compare Diophantus' in Arithmetica and Euclid's Elements as mathematical texts. 
	\end{enumerate} 
\end{enumerate}
\end{document}