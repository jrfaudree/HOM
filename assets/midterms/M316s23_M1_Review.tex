\documentclass[11pt,fleqn]{article} 
\usepackage[margin=0.8in, head=0.8in]{geometry} 
\usepackage{amsmath, amssymb, amsthm}
\usepackage{fancyhdr} 
\usepackage{palatino, url, multicol}
\usepackage{graphicx, pgfplots} 
\usepackage[all]{xy}
\usepackage{polynom} 
%\usepackage{pdfsync} %% I don't know why this messes up tabular column widths
\usepackage{enumerate}
\usepackage{framed}
\usepackage{setspace}
\usepackage{array,tikz}

\pgfplotsset{compat=1.6}

\pgfplotsset{soldot/.style={color=black,only marks,mark=*}} \pgfplotsset{holdot/.style={color=black,fill=white,only marks,mark=*}}
\pgfplotsset{my style/.append style={axis x line=middle, axis y line=
middle, xlabel={$x$}, ylabel={$y$} }}

%axis equal 
\pagestyle{fancy} 
\lfoot{}
\rfoot{Mid 1 Review}

\begin{document}
\renewcommand{\headrulewidth}{0pt}
\newcommand{\blank}[1]{\rule{#1}{0.75pt}}
\newcommand{\bc}{\begin{center}}
\newcommand{\ec}{\end{center}}
\renewcommand{\d}{\displaystyle}

\vspace*{-0.7in}

%%%%%%%%%intro page
\begin{center}
  \large
  \sc{Review for Midterm 1}\\
\end{center}
\noindent\textbf{Logistics}\\

Midterm 1 is Wednesday 15 February at our usual class time. (Distance students may need to adjust this.) It will be given in two parts. Part I will be taken first. It is intended to be short (10 minutes), closed-note, closed-book and you will know the questions in advance. Once you have completed Part I, you will turn it in and begin on Part II.  For Part II, you may use a calculator and bring two pages of notes. How much time you spend on Part I is up to you.  \\

\noindent\textbf{Book Sections}\\

The midterm will cover Chapters 1-2 and Chapter 3 Sections 1-3.\\

\noindent\textbf{Chapter-by-Chapter Summary}\\

Chapter 1: We learned a variety of ways of writing and recording numbers, including tally marks,  Peruvian quipus, Mayan symbols, ancient Greek, Egyptian, Babylonian and Chinese numerical systems. In order to describe some of the differences and similarities between systems, we learned terminology like base of a numerical system, whether the system is positional, additive, subtractive,  or ciphered. We discussed the materials used to write mathematics and the numerical system affected the mathematics itself. We practiced doing basic arithmetic (addition, subtraction, multiplication and division) using these systems.\\

Chapter 2: \\
Topics Discussed 
\begin{itemize}
\item Egyptian arithmetic including multiplication, division, representation of fractions, and the need for tables of fractions. 
\item The Rhind papyrus. Its history. Types of problems. The method of false position. 
\item Egyptian geometry. The nature of the problems and the solutions. Examples of correct and incorrect or approximate solutions.
\item Babylonian mathematics. The use of reciprocals for division. The consequences of a positional system without zero or sexagesimal point.
\item Babylonian solutions to problems reducing to quadratic equations.
\item Plimton 322. Its history and contents. 
\item Cairo Papyrus. Its history and contents. 
\item Methods of approximating square roots.\\
\end{itemize}

Chapter 3:\\
Topics Discussed:
\begin{itemize}
\item Thales of Miletos. His contributions to mathematics.
\item Pythagoras and the Pythagoreans. Their history, philosophy, and mathematics.
\item Figurative numbers. Algebraic proofs and proof-by-picture.
\item Zeno's paradox of Achilles and the tortoise.
\item Picture proofs of the Pythagorean Theorem. 
\item Incommensurable quantities. Their discovery and consequences.
\item Theon's approximations of roots.
\item Eudoxus' and his solution to the dilemma of incommensurable quantities.
\end{itemize}
\end{document}

