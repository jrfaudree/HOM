\documentclass[11pt,fleqn]{article} 
\usepackage[margin=0.8in, head=0.8in]{geometry} 
\usepackage{amsmath, amssymb, amsthm}
\usepackage{fancyhdr} 
\usepackage{palatino, url, multicol}
\usepackage{graphicx, pgfplots} 
\usepackage[all]{xy}
\usepackage{polynom} 
%\usepackage{pdfsync} %% I don't know why this messes up tabular column widths
\usepackage{enumerate}
\usepackage{framed}
\usepackage{setspace}
\usepackage{array,tikz}

\pgfplotsset{compat=1.6}

\pgfplotsset{soldot/.style={color=black,only marks,mark=*}} \pgfplotsset{holdot/.style={color=black,fill=white,only marks,mark=*}}
\pgfplotsset{my style/.append style={axis x line=middle, axis y line=
middle, xlabel={$x$}, ylabel={$y$} }}

%axis equal 
\pagestyle{fancy} 
\lfoot{}
\rfoot{Mid 3 Review}

\begin{document}
\renewcommand{\headrulewidth}{0pt}
\newcommand{\blank}[1]{\rule{#1}{0.75pt}}
\newcommand{\bc}{\begin{center}}
\newcommand{\ec}{\end{center}}
\renewcommand{\d}{\displaystyle}

\vspace*{-0.7in}

%%%%%%%%%intro page
\begin{center}
  \large
  \sc{Review for FE}\\
\end{center}
\noindent\textbf{Logistics}\\

The Final Exam is Thursday  May 4, 3:15-5:15 in our usual classroom. (Distance students may need to adjust this.) It will have one part, you may use a calculator and bring two pages of notes.  \\

\noindent\textbf{Book Sections}\\

The midterm will focus on the topics after Midterm III.\\

\noindent\textbf{Chapter-by-Chapter Summary}\\

Chapter 1: We learned a variety of ways of writing and recording numbers, including tally marks,  Peruvian quipus, Mayan symbols, ancient Greek, Egyptian, Babylonian and Chinese numerical systems. In order to describe some of the differences and similarities between systems, we learned terminology like base of a numerical system, whether the system is positional, additive, subtractive,  or ciphered. We discussed the materials used to write mathematics and the numerical system affected the mathematics itself. We practiced doing basic arithmetic (addition, subtraction, multiplication and division) using these systems.\\

Chapter 2: \\
Topics Discussed 
\begin{itemize}
\item Egyptian arithmetic including multiplication, division, representation of fractions, and the need for tables of fractions. 
\item The Rhind papyrus. Its history. Types of problems. The method of false position. 
\item Egyptian geometry. The nature of the problems and the solutions. Examples of correct and incorrect or approximate solutions.
\item Babylonian mathematics. The use of reciprocals for division. The consequences of a positional system without zero or sexagesimal point.
\item Babylonian solutions to problems reducing to quadratic equations.
\item Plimton 322. Its history and contents. 
\item Cairo Papyrus. Its history and contents. 
\item Methods of approximating square roots.\\
\end{itemize}

Chapter 3:\\
Topics Discussed:
\begin{itemize}
\item Thales of Miletos. His contributions to mathematics.
\item Pythagoras and the Pythagoreans. Their history, philosophy, and mathematics.
\item Figurative numbers. Algebraic proofs and proof-by-picture.
\item Zeno's paradox of Achilles and the tortoise.
\item Picture proofs of the Pythagorean Theorem. 
\item Incommensurable quantities. Their discovery and consequences.
\item Theon's approximations of roots.
\item Eudoxus' and his solution to the dilemma of incommensurable quantities.\\
\end{itemize}

Additional Topics since Midterm I.\\

\begin{itemize}
\item Hippocrates and the quadrature of a lune, in the context of his progress on two and the three construction problems from antiquity (squaring the circle and doubling the cube).
\item We also added context to the three construction problems from antiquity by understanding that doubling or trisecting many figures is easy.
\item Hippias and the Quadratrix. We learned the definition of this curve, its significance both as a curve and its role in trisecting an angle.
\end{itemize}

Chapter 4:\\

\begin{itemize}
\item Euclid and \textit{Elements of Geometry}. We learned about
	\begin{itemize}
	\item political and geographic factors that lead to the rise of the Museum in Alexandria.
	\item the history of Euclid's writing including \textit{Data}, \textit{Conic Sections}, and \textit{Porisms}
	\item the history of \textit{Elements} as a document
	\item the structure of Book I of \textit{Elements}, its contents including several propositions with proofs, and its historical importance.
	\item the importance of Postulate 5 (or the parallel postulate or the 5th axiom).
	\item Euclid's proof of the Pythagorean Theorem and its converse.
	\item the nature of Book II concerning geometric algebra
	\item Eudoxus' theory of proportion appearing in Book V
	\item Euclid's number theory appearing in Books VII, VIII, and IX including its structure, some more prominent results with proofs, and its historical importance.
	\item the Euclidean algorithm for finding the greatest common divisor of two positive integers.
	\end{itemize}
\item Eratosthenes, his device for doubling the cube, his scheme for estimating the circumference of the earth, his sieve for identifying prime numbers
\item The nature of and historical importance of Ptolemy's \textit{Almagest}.
\item Archimedes. We learned about
	\begin{itemize}
	\item many interesting aspects of his life and his contributions to mathematics and science.
	\item his strategy for approximating $\pi$.
	\item his formula for the area of a circle in terms of its radius and circumference.
	\item the topics in his work \textit{The Sand-Reckoner} and \textit{On Spirals}
	\item his quadrature of a parabolic segment
	\end{itemize}
\end{itemize}

Additional Topics since Midterm II.\\

Chapter 5:\\
\begin{itemize}
\item (Section 5.1, 5.2) Diophantus, his use of symbols, \emph{Arithmetica}, solutions to equations
\item (Section 5.3) Diophantine equations in Greece, India and China. We learned about complete solutions to linear diophantine equations.
\item (Section 5.4) We learned about Hypatia and the general decline of Alexandria as the center of the mathematical world.
\item (Section 5.5) We learned about the mathematics of the Islamic empire from the algebra of al Khwarizmi to Omar Al Khayyam to al Tusi.  We also learned about Chinese mathematics especially Liu Hui (\emph{Nine Chapters} and \emph{Sea Island Mathematical Manual}), solutions to systems of linear equations and the use of negative numbers. 
\end{itemize}

Chapter 6:\\
\begin{itemize}
\item The role of Hindu and Islamic mathematicians with the development of a base 10, positional, 10-symbol method of numerical representation. The role of Leonardo of Pisa (Fibonacci) in the introduction of this system to Europe.
\end{itemize}

Chapter 7:\\
\begin{itemize}
\item Solutions to cubic equations and the contributions of Fra Luca Pacioli, Nicolo Tartaglia, and Girolamo Cardano. The strategy of the reduced cubic.
\item Cardano's \emph{Ars Magna}, Cardano's formula,  its role in Rafael Bombelli's work with imaginary numbers.
\item Ludovico Ferrari's solution to quartic equations and the quest for a formula to solve quintic equations including the contributions of Paolo Ruffini and Niels Henrik Abel.
\end{itemize}

Chapter 8:\\
\begin{itemize}
\item We talked about the evolution of astronomy including contributions of Nicolaus Copernicus, Galileo Galilei, Johannes Kepler, Tycho Brahe and Isaac Newton.
\item We thought through the evolution of algebraic notation including Diophantus, Cardano and Bombelli, Francois Vieta, Robert Recorde, Christoff Rudolff, Michael Stifel, Rene Descartes.
\item We see improved computation via the decimal fractions of Simon Stevin and logarithms of John Napier.
\item We looked at the start of coordinate geometry due to Rene Descartes and Pierre de Fermat.
\item We looked at the calculus of Newton and Leibniz.
\item We see repeatedly the role of publication (or lack thereof), the style of publication, and translations in the transition and development of new mathematics.
\end{itemize}

Chapter 9:\\
\begin{itemize}
\item The motivation and development of probability theory. 
\item The role of John Graunt, Girolamo Cardano, Pierre de Fermat, Blaise Pascal, and Antoine Gombaud the Chevalier de Mere in this process.
\end{itemize}

Chapter 10:\\
\begin{itemize}
\item A brief look at Leonhard Euler and Carl Friedrich Gauss
\end{itemize}

Chapter 11:\\
\begin{itemize}
\item Non-euclidean geometry.
\item The role of Euclid's 5th Postulate in the development of non-Euclidean geometry.
\item The long history of skepticism about the role of Euclid's 5th Postulate in Euclidean geometry including contributions of Thabit ibn Qurra,  Omar Khayyam, Girolamo Saccheri, Johann Lambert, Adrien-Marie Legendre.
\item The development of hyperbolic geometry by Carl Gauss, Janos Bolyai, and Nicolai Ivanovich Lobechevsky. The development of spherical geometry by Bernhard Riemann.
\end{itemize}
\end{document}

