\documentclass[11pt,fleqn]{article} 
\usepackage[margin=0.8in, head=0.8in]{geometry} 
\usepackage{amsmath, amssymb, amsthm}
\usepackage{fancyhdr} 
\usepackage{palatino, url, multicol}
\usepackage{graphicx, pgfplots} 
\usepackage[all]{xy}
\usepackage{polynom} 
%\usepackage{pdfsync} %% I don't know why this messes up tabular column widths
\usepackage{enumerate}
\usepackage{framed}
\usepackage{setspace}
\usepackage{array,tikz}

\pgfplotsset{compat=1.6}

\pgfplotsset{soldot/.style={color=black,only marks,mark=*}} \pgfplotsset{holdot/.style={color=black,fill=white,only marks,mark=*}}
\pgfplotsset{my style/.append style={axis x line=middle, axis y line=
middle, xlabel={$x$}, ylabel={$y$} }}

%axis equal 
\pagestyle{fancy} 
\lfoot{}
\rfoot{Mid 1 Review}

\begin{document}
\renewcommand{\headrulewidth}{0pt}
\newcommand{\blank}[1]{\rule{#1}{0.75pt}}
\newcommand{\bc}{\begin{center}}
\newcommand{\ec}{\end{center}}
\renewcommand{\d}{\displaystyle}

\vspace*{-0.7in}

%%%%%%%%%intro page
\begin{center}
  \large
  \sc{Review for Midterm 2}\\
\end{center}
\noindent\textbf{Logistics}\\

Midterm 2 is Friday 24 March at our usual class time. (Distance students may need to adjust this.) It will be given in two parts. Part I will be taken first. It is intended to be short (10 minutes), closed-note, closed-book and you will know the questions in advance. Once you have completed Part I, you will turn it in and begin on Part II.  For Part II, you may use a calculator and bring two pages of notes. How much time you spend on Part I is up to you.  \\

\noindent\textbf{Book Sections}\\

The midterm will cover Chapters 1-2 and Chapter 3 Sections 1-3 along with Chapter 3 Sections 3 and 4 and Chapter 4 Sections 1-5. Note that the mathematics problems will be limited to those in homework assignments 4, 5, and 6.\\

\noindent\textbf{Chapter-by-Chapter Summary}\\

Chapter 1: We learned a variety of ways of writing and recording numbers, including tally marks,  Peruvian quipus, Mayan symbols, ancient Greek, Egyptian, Babylonian and Chinese numerical systems. In order to describe some of the differences and similarities between systems, we learned terminology like base of a numerical system, whether the system is positional, additive, subtractive,  or ciphered. We discussed the materials used to write mathematics and the numerical system affected the mathematics itself. We practiced doing basic arithmetic (addition, subtraction, multiplication and division) using these systems.\\

Chapter 2: \\
Topics Discussed 
\begin{itemize}
\item Egyptian arithmetic including multiplication, division, representation of fractions, and the need for tables of fractions. 
\item The Rhind papyrus. Its history. Types of problems. The method of false position. 
\item Egyptian geometry. The nature of the problems and the solutions. Examples of correct and incorrect or approximate solutions.
\item Babylonian mathematics. The use of reciprocals for division. The consequences of a positional system without zero or sexagesimal point.
\item Babylonian solutions to problems reducing to quadratic equations.
\item Plimton 322. Its history and contents. 
\item Cairo Papyrus. Its history and contents. 
\item Methods of approximating square roots.\\
\end{itemize}

Chapter 3:\\
Topics Discussed:
\begin{itemize}
\item Thales of Miletos. His contributions to mathematics.
\item Pythagoras and the Pythagoreans. Their history, philosophy, and mathematics.
\item Figurative numbers. Algebraic proofs and proof-by-picture.
\item Zeno's paradox of Achilles and the tortoise.
\item Picture proofs of the Pythagorean Theorem. 
\item Incommensurable quantities. Their discovery and consequences.
\item Theon's approximations of roots.
\item Eudoxus' and his solution to the dilemma of incommensurable quantities.\\
\end{itemize}

Addition Topics since Midterm I.\\

\begin{itemize}
\item Hippocrates and the quadrature of a lune, in the context of his progress on two and the three construction problems from antiquity (squaring the circle and doubling the cube).
\item We also added context to the three construction problems from antiquity by understanding that doubling or trisecting many figures is easy.
\item Hippias and the Quadratrix. We learned the definition of this curve, its significance both as a curve and its role in trisecting an angle.
\end{itemize}

Chapter 4:\\

\begin{itemize}
\item Euclid and \textit{Elements of Geometry}. We learned about
	\begin{itemize}
	\item political and geographic factors that lead to the rise of the Museum in Alexandria.
	\item the history of Euclid's writing including \textit{Data}, \textit{Conic Sections}, and \textit{Porisms}
	\item the history of \textit{Elements} as a document
	\item the structure of Book I of \textit{Elements}, its contents including several propositions with proofs, and its historical importance.
	\item the importance of Postulate 5 (or the parallel postulate or the 5th axiom).
	\item Euclid's proof of the Pythagorean Theorem and its converse.
	\item the nature of Book II concerning geometric algebra
	\item Eudoxus' theory of proportion appearing in Book V
	\item Euclid's number theory appearing in Books VII, VIII, and IX including its structure, some more prominent results with proofs, and its historical importance.
	\item the Euclidean algorithm for finding the greatest common divisor of two positive integers.
	\end{itemize}
\item Eratosthenes, his device for doubling the cube, his scheme for estimating the circumference of the earth, his sieve for identifying prime numbers
\item The nature of and historical importance of Ptolemy's \textit{Almagest}.
\item Archimedes. We learned about
	\begin{itemize}
	\item many interesting aspects of his life and his contributions to mathematics and science.
	\item his strategy for approximating $\pi$.
	\item his formula for the area of a circle in terms of its radius and circumference.
	\item the topics in his work \textit{The Sand-Reckoner} and \textit{On Spirals}
	\item his quadrature of a parabolic segment
	\end{itemize}
\end{itemize}
\end{document}

