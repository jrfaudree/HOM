\documentclass[12pt]{article}
\usepackage[top=1in, bottom=1in, left=.75in, right=.75in]{geometry}
\usepackage{amsmath, enumerate}
\usepackage{fancyhdr}
\usepackage{graphicx, xcolor, setspace, array}
\usepackage{txfonts}
\usepackage{multicol,coordsys,pgfplots}
\usepackage[scaled=0.86]{helvet}
\renewcommand{\emph}[1]{\textsf{\textbf{#1}}}
\usepackage{anyfontsize}
\usepackage{tikz,pgfplots}
\usetikzlibrary{calc,arrows.meta}
\pgfplotsset{compat = newest}

\parindent 0pt
\parskip 4pt
\pagestyle{fancy}
\fancyfoot[C]{\emph{\thepage}}
\fancyhead[L]{\ifnum \value{page} > 1\relax\emph{Math 316: Exam 2}\fi}
\fancyhead[R]{\ifnum \value{page} > 1\relax\emph{Spring 2025}\fi}
\headheight 12pt
\renewcommand{\headrulewidth}{0pt}
\renewcommand{\footrulewidth}{0pt}
\let\ds\displaystyle

\newcommand{\be}{\begin{enumerate}}
\newcommand{\ee}{\end{enumerate}}

\begin{document}
%% Front Page
{\emph{\fontsize{20}{20}\selectfont Spring 2025 \hfill
\hfill Math 316}}

\begin{center}
{\emph{{\fontsize{20}{20}\selectfont Exam 2}
}}
\end{center}

\strut\vtop{\halign{\emph#\hskip 0.5em\hfil&#\hbox to 2in{\hrulefill}\cr
\emph{\fontsize{18}{22}\selectfont Name:}&\cr
\hfill
\cr}}

{\fontsize{18}{22}\selectfont\emph{Rules:}}

\begin{itemize}
\item Partial credit will be awarded, but you must show your work.

\item No notes, books, or cell phones are allowed.

\item Calculators are allowed. 

\end{itemize}

\fbox{NOTE:} The exam is formatted in order to provide plenty of space. A complete answer does \emph{not} need to fill the given space. Indeed, it is not necessary.\\




\vfill
\def\emptybox{\hbox to 2em{\vrule height 16pt depth 8pt width 0pt\hfil}}
\def\tline{\noalign{\hrule}}
\centerline{\vbox{\offinterlineskip
{
\bf\sf\fontsize{18pt}{22pt}\selectfont
\hrule
\halign{
\vrule#&\strut\quad\hfil#\hfil\quad&\vrule#&\quad\hfil#\hfil\quad
&\vrule#&\quad\hfil#\hfil\quad&\vrule#\cr
height 3pt&\omit&&\omit&&\omit&\cr
&Problem&&Possible&&Score&\cr\tline
height 3pt&\omit&&\omit&&\omit&\cr
&1&&15 + 2 extra credit &&\emptybox&\cr\tline
&2&&15	&&\emptybox&\cr\tline
&3&&20 + 2 extra credit	&&\emptybox&\cr\tline
&4&&20	&&\emptybox&\cr\tline
&5&&15	&&\emptybox&\cr\tline
&6&&15	&&\emptybox&\cr\tline
&Total&&100&&\emptybox&\cr
}\hrule}}}

\vfill
\newpage
\be
%% Arabic Quad
\item (15 points) Solve the quadratic equation below using the Arabic method of completing the square. Include the algebraic steps and the accompanying figure.\\

$$\text{equation:} \:\:\: x^2+12x=10$$

\vfill
\textbf{Extra Credit (2 points):} Describe two ways in which the solution above is different from the modern method of completing the square.\\
\vspace{2in}
\newpage
%%Khayyam
\item (15 points) Describe Omar Khayyam's contributions to algebra, approximately when this occurred, how his approach is similar and different from our modern approach.
	\be
	\item approximate date:
	\vspace{1in}
	\item description of contributions:
	\vfill
	\item similar to a modern approach:
	\vfill
	\item different from a modern approach:
	\vfill
	\ee
\newpage
%%Reduced cubic
\item (20 points) 
	\be
	\item Explain what is meant by a \emph{reduced cubic}.
	\vspace{1in}
	\item Find the reduced cubic for $x^3+6x^2+x=1.$ 
	\vfill
	\item What was the importance of the reduced cubic to Girolamo Cardan? 
	\vfill
	\item Describe two ways in which the Cardan's formulas for solving cubic equations catalyzed further mathematical research. Include names, dates and mathematical topic/result.
	\vfill
	\ee
\newpage
%%% 4-lines and descartes
\item (20 points)
	\be
	\item Find the locus of points, $P(x,y),$ satisfying the four line problem where the lines are
	$$L_1: x=0,\: L_2: y=0,\: L_3: x=5,\: L_4: y=10,$$
	$d_i$ is the distance between $P$ and $L_i$, and the constraint is $d_1d_2=d_3d_4.$
	
	
	
	\vfill
	\item Find two specific points in the locus.
	\vspace{0.5in}
	\item These ``$n$ line problems" (4 line problem, 5 line problem, etc) appear in Ren\'{e} Descartes' La G\'{e}om\'{e}trie. Why? What was Descartes attempting to demonstrate?
	\vfill
	\item \textbf{extra credit (2 points)} Describe in words or by a graph what the locus of points from part (a).
	\vspace{1in}
	\ee
\newpage
%%calculus
\item (15 points) Give an example of a mathematician who solved a problem we could consider a typical calculus 1 problem prior to 1700. Describe their approach and compare it to our modern approach to the problem.
	\be
	\item Name of mathematician and approximate dates:
	\vspace{1in}
	\item Description of calculus problem:
	\vfill
	\item Their method/strategy of solution:
	\vfill
	\item Similarities:
	\vfill
	\item Differences:
	\vfill
	\ee
\newpage
\item (15 points) List five significant mathematical contributions of Isaac Newton and/or Gottfried Leibniz. Indicate which of the two discovered/developed the concept. Make sure you include \textbf{at least one contribution from each person}.
	\be
	\item[(1):] \quad \\
	
	\vfill
	\item[(2):] \quad \\

	\vfill
	\item[(3):] \quad \\

	\vfill
	\item[(4):] \quad \\

	\vfill
	\item[(5):] \quad \\

	\vfill
	\ee
	
\ee


\end{document}

%%%%ENDDOCUMENT


