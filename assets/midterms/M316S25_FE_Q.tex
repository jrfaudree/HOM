% !TEX TS-program = pdflatexmk
\documentclass[12pt]{article}

% Layout.
\usepackage[top=1in, bottom=0.75in, left=1in, right=1in, headheight=1in, headsep=6pt]{geometry}

% Fonts.
\usepackage{mathptmx}
\usepackage[scaled=0.86]{helvet}
\renewcommand{\emph}[1]{\textsf{\textbf{#1}}}

% Misc packages.
\usepackage{amsmath,amssymb,latexsym}
\usepackage{graphicx,hyperref}
\usepackage{array}
\usepackage{xcolor}
\usepackage{multicol}
\usepackage{tabularx,colortbl}
\usepackage{enumitem}

\usepackage{fancyhdr}
\pagestyle{fancy} 
\lhead{\large\sf\textbf{MATH 316: History of Math}}
\rhead{\large\sf\textbf{Final Exam Questions}}

\begin{document}
\noindent Basic Information\\
The final exam is cumulative. It is 10:15am-12:15pm Friday May 2. The list of questions here is \emph{in addition} to those questions. All questions from Midterm I and Midterm II are included.  \\

\begin{enumerate}
\item Pick one of the following topics and describe its evolution over time. Your description must include at least 4 points in time. To earn full credit the span must roughly include the span of time we discussed in our class (3000BC - 1800's) and must be specific. I have given an example for the topic probability.\\
Topics you may choose: numerical representation, the concept of number, algebra, calculus.

One answer.
Intuitive notions of probability are as old as humans and their games of chance. As early as 3000BC there are examples of astragali or dice. Around 700's AD there are examples of Arabic mathematicians, such as al-Khalil, who developed the counting techniques (combinations and permutations) and frequency analyses, a particular application of probability. While Cardan wrote one of the first texts on probability associated with gambling in the 1500s, the foundational strategies and definitions appear in the 1600's. In attempting to solve the Problem of Points, Fermat and Pascal develop the modern definitions of the probability of an event and proto-definitons of expected value.

\item Given a simple problem of points, solve it using Fermat's approach and Pascal's approach.

\item Describe roughly when and where Leonhard Euler lived and at least three distinct important mathematical contributions he made.

\item Describe roughly when and where Carl Friedrich Gauss lived and at least three distinct important mathematical contributions he made. 

\item Describe the role of Euclid's Elements in the development of non-Euclidean geometry. 

\item Describe the process by which non-Euclidean geometry is developed. Include names and dates.

\item Describe three ways in which Euclidean geometry and hyperbolic (or elliptical geometry) are different.

\item Give a least three distinct examples of mathematical topics that were developed/discovered independently by two different mathematician at the same time. A complete answer includes the mathematical topic, the two co-discoverers and the approximate time period. Full credit only if all the mathematicians are different.

\item In a typical introductory Calculus I course, the concepts of limits, derivatives, and integrals are presented in that order -- first limits, second derivatives and their applications, and third integrals along with their applications. Compare this to how these concepts were first explored and/or discovered by mathematicians. Be specific about mathematicians, mathematical topic, and timeframe.
\end{enumerate}
\end{document}