% !TEX TS-program = pdflatexmk
\documentclass[11pt]{article}
\usepackage[margin=.8in]{geometry}
\usepackage{amsmath,amssymb,amsthm, latexsym, mathrsfs, pdfsync, multicol,
fancybox, fancyhdr,
graphicx, enumerate,
subfig, tikz, pgfplots,array}

%\singlespacing
\def\RR{{\mathbb R}}
\def\NN{{\mathbb N}}
\def\ZZ{{\mathbb Z}}
\def\QQ{{\mathbb Q}}
\def\CC{{\mathbb C}}
\def\bc{\begin{center}}
\def\ec{\end{center}}
\def\be{\begin{enumerate}}
\def\ee{\end{enumerate}}
\def\bi{\begin{itemize}}
\def\ei{\end{itemize}}
\def\t{\times}
\newcommand{\ol}[1]{\overline{#1}}
\newcommand{\oimp}[1]{\overset{#1}{\Longleftrightarrow}}
\newcommand{\bv}[1]{\ensuremath{ \mathbf{\vec{#1}}} }
\renewcommand{\d}{\displaystyle}
\newcommand{\blank}[1]{\rule{#1}{0.75pt}}

\usetikzlibrary{calc}

\lhead{\sc{Math 316 Hist. of Math.}}
\chead{\large \sc Midterm III} 
\rhead{\sc Spring 2023}
\cfoot{}
\pagestyle{fancy}
%
\begin{document}
\thispagestyle{fancy}

\vspace{1in}

\textbf{Student Name:}\hspace{2in}

\vspace{1in}

{
\renewcommand{\baselinestretch}{1.8}
\setlength{\tabcolsep}{.2in}
\normalsize
\begin{center}
\begin{tabular}{|c|c|c|c|}
\hline
&Problem&Total Points&\parbox{.8in}{\hfil Score\hfil}\\
\hline
Part I&&24&\\
\hline
Part II &&&\\
\hline
&1&16&\\
\hline
&2&16&\\
\hline
&3&14&\\
\hline
&4&30&\\
\hline
\hline
Total&100&&\\
\hline
\end{tabular}

\end{center}
}

\vspace{1in} 

Guidelines
\begin{itemize}
\item You have 1 hour to take the exam.
\item The exam will be given in two parts.
\item Part I is written without any aids: no notes, no book, no phone. You should spend no more than 15 minutes on Part I. 
\item Return your completed Part I to the proctor and you will be given Part II. You cannot go back to Part I once you have turned it in.
\item For Part II, you may use a calculator and two pages of notes (i.e. two sheets of paper with writing on both sides of each sheet).
\end{itemize}

\newpage


\vspace*{-0.3in}

\bc Part I \ec

This part is written without notes or aids of any kind. It is worth 24 points out of 100 total points. \\

Below is a list of 6 mathematicians. For each mathematician, give approximate dates, approximate locations, and two specific mathematical contributions. \\

Once you have completed Part I and turn it in, you will be given Part II. You cannot return to Part I once it has been turned in.\\

\be
\item Mohammed al-Khowarizmi
\vfill
\item Girolamo Cardano
\vfill
\item Diophantus
\vfill
\item Pierre de Fermat
\vfill
\item Nicolai Lobechevsky
\vfill
\item Isaac Newton
\vfill
\ee
\newpage
\bc Part II \ec

This part can be written with the aid of a calculator and two pages of notes with writing on the front and back of the paper. It is worth 76 points out of 100 points total. \\

\be
%%%Diophantus
\item (16 points)
	\be
	\item Below is a quote from Thomas Heath's 1910 translation of Diophantus' \emph{Arithmetica}. Rewrite the problem and Diophantus' solution in your own words using modern notation.
	
	\begin{quote} (Book II, Problem 10) To find two square numbers having a given difference.\\
	Given difference 60.\\
	Side of one number $x$, side of the other $x$ plus any number the square of which is not \\
	\quad \vspace{.2in}\quad \quad greater than 60, say 3.\\
	Therefore $(x+3)^2-x^2=60;$\\
	$x=8\frac{1}{2}$, and \\
	\quad \vspace{.2in}\quad \quad the required squares are $72\frac{1}{4}, \: 132\frac{1}{4}$. \end{quote}
	\vfill
	\item Demonstrate that this strategy described will work \emph{in general} by replacing the difference, $60,$ with $d$ and replacing the amount added, 3, with $a.$ 
		\vfill
	\item Since Diophantus is clearly looking for two numbers (his answer is ``...the required squares are $72\frac{1}{4}, \: 132\frac{1}{4}$"), why doesn't he choose $x$ to be one number and $y$ to be the second?
	\vspace{1in}
	\ee
\newpage
%%%% Khayyam
\item (16 points)
	\be
	\item Show that one can solve $x^3+d=cx$ by intersecting the hyperbola $y^2-x^2+\frac{d}{c}x=0$ with the parabola $x^2=\sqrt{c} y.$
	\vfill
	\item Explain how Omar Khayyam's solutions to cubic equations were different from our modern view of solutions to cubic equations.
	\vfill
	\item Find positive, integer values for $c$ and $d$ such that the only real solution to  $x^3+d=cx$ is negative.	(Hint: You might find it easier to work with $f(x)=x^3-cx+d.$)
	\vfill
	\item Explain how Khayyam would have viewed this solution?
	\vspace{1in}
	\ee
\newpage
\item (14 points)
	\be
	\item Show how to find the reduced version of $x^3+21x=9x^2+5$ using the substitution $x=y+3.$
	\vfill
	\item Why would Cardano (and others) transform cubic equations into reduced cubic equations as opposed to just solving them in their original form?
	\vfill
	\item Explain how Cardano's formula led to the beginning of the arithmetic of complex numbers.
	\vfill
	\ee
\newpage
\item (30 points) (Short Answer)
\be
	\item Explain how the coordinate geometry of Rene Descartes and Pierre de Fermat was different then our modern version.
	\vfill
	\item Why are Isaac Newton and Gottfried Leibniz credited with inventing/discovering Calculus when we know mathematicians were solving area-under-the-curve and tangent line problems for hundreds of years prior?
	\vfill
	\item Describe what motivates early development of probability and statistics. Include specific people.
	\vfill
	\newpage
	\item Describe the role of Euclid's 5th axiom in the development of non-euclidean geometry.
	\vfill
	\item Typically, researchers (in mathematics and other disciplines) aggressively publish their findings. Give an example of a mathematician who did not rush to publish their work and explain why. 
	\vfill
	\item Describe three contributions to mathematics attributed to Leonhard Euler.
	\vfill
	\ee
\ee
\end{document}
%%%%%%%%%%%%%%%%%%%%%%%%%%
%%%%%%%END
%%%%%%%%%%%%%%%%%%%%%%%%%%


 