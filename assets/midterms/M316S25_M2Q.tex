% !TEX TS-program = pdflatexmk
\documentclass[12pt]{article}

% Layout.
\usepackage[top=1in, bottom=0.75in, left=1in, right=1in, headheight=1in, headsep=6pt]{geometry}

% Fonts.
\usepackage{mathptmx}
\usepackage[scaled=0.86]{helvet}
\renewcommand{\emph}[1]{\textsf{\textbf{#1}}}

% Misc packages.
\usepackage{amsmath,amssymb,latexsym}
\usepackage{graphicx,hyperref}
\usepackage{array}
\usepackage{xcolor}
\usepackage{multicol}
\usepackage{tabularx,colortbl}
\usepackage{enumitem}

\usepackage{fancyhdr}
\pagestyle{fancy} 
\lhead{\large\sf\textbf{MATH 316: History of Math}}
\rhead{\large\sf\textbf{Midterm II Questions}}

\begin{document}
Basic Information\\
The focus of this midterm is the material since the first midterm. This includes the non-Greek discussion in Chapter 5, Chapters 6, 7, and 8. \\

\begin{enumerate}
\item (Like \S 5.5 \#1) Be able to solve a quadratic equation by the Arabic method of completing the square and be able to draw the accompanying picture.
\item Describe Muhammad al-Khwarizimi's contribution to \textbf{algebra} including approximate dates and an explanation of how his approach/solution was similar to and different from our own.
\item The same question as \#2 but replace with Omar Khayyam.
\item Be able to solve some typical word problems as they appear in \textit{Nine Chapters} or al-Khwarizmi's \textit{Algebra}, or Fibonacci's \textit{Liber Abaci}. (See homework 6.) 
\item Describe how our modern base 10 positional system with 10 symbols comes to be adopted in Western Europe by describing two specific moments in this process. Include specific people and dates.
\item Describe how our modern algebraic notation is developed by describing two specific moments in this process. Include specific people and dates.
\item Give at least two specific examples of Calculus formulas, problem solutions or strategies that were known prior to either Newton's or Leibniz's development of Calculus. A complete answer includes the mathematical idea, the mathematician who conceived it, and an approximate date.
\item Know what is meant by a \textit{reduced cubic}, how to find it, and why it is important historically.
\item Be able to solve a problem of the form: Describe the locus of a set of points satisfying a particular 3 or 4 line problem.
\item Describe some of the topics in Descartes' \textit{La Geometrie} and describe its importance to the development of mathematics generally and to calculus specifically.
\item Explain how Girolamo Cardan's solution to the cubic is similar to and different from the modern one. Elaborate on how Cardan's work catalyzed further mathematical research. Name specific mathematicians, dates, and mathematical topics.
\item Give an example of a mathematician who solved a problem (or type of problem) from Calculus prior to 1700 and explain how this solution relates to our modern solution. Your answer should include both similarities and differences.
\item Give a brief description of at least two mathematical contributions of Isaac Newton, including an approximate time in which is was developed or published.
\item Give a brief description of at least two mathematical contributions of Gottfried Leibniz, including an approximate time in which is was developed or published.
\item Give two examples of the importance of mathematicians who made translations \textbf{with elaboration or expansions} of original mathematical research. A complete answer includes the original work, the expanded work, and its significance.
\item Give two examples of the enduring influence of ancient Greek mathematics on the mathematics of Europe in the 1500's-1600's.
\item Describe one idea, story, mathematical approach, development from this class that has surprised or delighted you or that you expect you will remember or contemplate after the course is over.
\end{enumerate}
\end{document}