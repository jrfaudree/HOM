\documentclass[12pt]{article}
\usepackage[top=1in, bottom=1in, left=.75in, right=.75in]{geometry}
\usepackage{amsmath, enumerate}
\usepackage{fancyhdr}
\usepackage{graphicx, xcolor, setspace, array}
\usepackage{txfonts}
\usepackage{multicol,coordsys,pgfplots}
\usepackage[scaled=0.86]{helvet}
\renewcommand{\emph}[1]{\textsf{\textbf{#1}}}
\usepackage{anyfontsize}
\usepackage{tikz,pgfplots}
\usetikzlibrary{calc,arrows.meta}
\pgfplotsset{compat = newest}

\parindent 0pt
\parskip 4pt
\pagestyle{fancy}
\fancyfoot[C]{\emph{\thepage}}
\fancyhead[L]{\ifnum \value{page} > 1\relax\emph{Math 316: Exam 1}\fi}
\fancyhead[R]{\ifnum \value{page} > 1\relax\emph{Spring 2025}\fi}
\headheight 12pt
\renewcommand{\headrulewidth}{0pt}
\renewcommand{\footrulewidth}{0pt}
\let\ds\displaystyle

\newcommand{\be}{\begin{enumerate}}
\newcommand{\ee}{\end{enumerate}}

\begin{document}
%% Front Page
{\emph{\fontsize{20}{20}\selectfont Spring 2025 \hfill
\hfill Math 316}}

\begin{center}
{\emph{{\fontsize{20}{20}\selectfont Exam 1}
}}
\end{center}

\strut\vtop{\halign{\emph#\hskip 0.5em\hfil&#\hbox to 2in{\hrulefill}\cr
\emph{\fontsize{18}{22}\selectfont Name:}&\cr
\hfill
\cr}}

{\fontsize{18}{22}\selectfont\emph{Rules:}}

\begin{itemize}
\item Partial credit will be awarded, but you must show your work.

\item No notes, books, or cell phones are allowed.

\item Calculators are allowed. 

\end{itemize}

\fbox{NOTE:} The exam is formatted in order to provide plenty of space. A complete answer does \emph{not} need to fill the given space. Indeed, it is not necessary.\\




\vfill
\def\emptybox{\hbox to 2em{\vrule height 16pt depth 8pt width 0pt\hfil}}
\def\tline{\noalign{\hrule}}
\centerline{\vbox{\offinterlineskip
{
\bf\sf\fontsize{18pt}{22pt}\selectfont
\hrule
\halign{
\vrule#&\strut\quad\hfil#\hfil\quad&\vrule#&\quad\hfil#\hfil\quad
&\vrule#&\quad\hfil#\hfil\quad&\vrule#\cr
height 3pt&\omit&&\omit&&\omit&\cr
&Problem&&Possible&&Score&\cr\tline
height 3pt&\omit&&\omit&&\omit&\cr
&1&&15 &&\emptybox&\cr\tline
&2&&15	&&\emptybox&\cr\tline
&3&&20	&&\emptybox&\cr\tline
&4&&20	&&\emptybox&\cr\tline
&5&&15	&&\emptybox&\cr\tline
&6&&15	&&\emptybox&\cr\tline
&Extra Credit&&(5)&&\emptybox&\cr\tline
&Total&&100&&\emptybox&\cr
}\hrule}}}

\vfill
\newpage
\be
%%Problem 1 Ancient Egyptians
\item (15 points) Use ancient Egyptian methods to complete the following calculations.
	\begin{enumerate}
	\item Multiply 50 by 42.
	\vfill
	\item Divide 159 by 42.
	\vfill
	\item Show that unit fraction decomposition is not unique by giving an example of a rational number $q$ and two different unit fraction decompositions of $q$ (where the unit fractions must be distinct).
	\vfill
	\ee
\newpage
%%Problem 2 Ancient Babylonians
\item (15 points) This problem concerns the base-60 numerical representation of the ancient Babylonians. We will use our textbook's notation.
	\be
	\item Write the base 60 number $3;45$ in base 10.
	\vfill
	\item Explain, with computation, how $3;45$ can be the reciprocal of 16.
	\vfill
	\item Describe two advantages of the Babylonian base 60 representation of numbers when compared to ancient Egyptian representation.
	\vfill
	\ee
\newpage
%%Problem 3 Incommensurability
\item (20 points) This question is about the idea of \textbf{incommensurable quantities.} 
	\begin{enumerate}
	\item What did ancient Greek mathematicians mean by the words ``line segments AB and BC are incommensurable"?
	\vfill
	\item How, when, and who discovered the existence of incommensurable line segments.
	\vfill 
	\item Describe (with pictures) how ancient Greek mathematicians could have constructed a line segment of length $\sqrt{n}$ for any positive integer $n.$
	\vfill
	\item Why is the existence of incommensurable quantities a crisis for Greek mathematicians?
	\vfill
	\end{enumerate}
\newpage
%%Problem 4 Euclid's Elements
 \item (20 points) This question is about Euclid's \emph{Elements}.
	\begin{enumerate}
	\item When and where was it written?\\
	\vspace{.5in}
	\item Describe its contents including its structure and its style of exposition.\\
	\vfill
	
	\item State the 5th Postulate and explain how it is different from the other four postulates.\\
	\vfill
	\item Compare Euclid's proof of the infinitude of primes from a modern one.  \\
	\vfill
	\item State two notable results found in Euclid's \emph{Elements} (other than the proposition that there are an infinite number of primes.)
	\vfill
	\end{enumerate}
\newpage
%%Problem 5  quaddrature
\item (15 points) This question is concerns the notion of the quadrature of geometric figures.
	\begin{enumerate}
	\item Explain the what is meant by the statement that Hippocrates accomplished the quadrature of a lune and explain why this was important. (You may want to draw pictures to illustrate your words.)\\
	\vfill
	\item Describe the method Archimedes used in his quadrature of a parabolic segment. (You may want to draw pictures to illustrate your words.)\\
	\ee
	\vfill
\newpage
%%problem 6 short answer
\item (15 points) Short answer.
	\be
	\item Explain how Archimedes' statement that \emph{the area of a circle is to the square on its diameter as 11 is to 14} can be interpreted as an estimation of $\pi$.
	\vfill
	\item Describe two ways in which Diophantus' notation in \emph{Arithmetica} was novel and describe two of its limitations.
	\vfill
	\item  Give one example of cut-and-paste algebra. A complete answer includes a picture (or pictures) and a modern algebraic identity that the picture (or pictures) demonstrates.
	\vfill
	
	\end{enumerate}
\ee
\newpage
%%Extra Credit
\textbf{Extra Credit:} (5 points) Below is Problem 27 from Book I from Diophantus' \emph{Arithemetica} as it appears in Thomas Heath's 1910 translation. Provide a modern algebraic explanation of the problem and the solution. Full points for explaining the trick Diophantus is employing that make his scheme work.\\

\begin{quote}
To find two numbers such that their sum and product are given numbers.\\

Given sum 20, given product 96.\\
2x the difference of the required numbers.\\
Therefore the numbers are $10+x,$ $10-x.$\\
Hence, $100-x^2=96.$\\
Therefore, $x=2,$ and \\
\quad \hspace{1cm} the required numbers are 12, 8.
\end{quote}


\end{document}

%%%%ENDDOCUMENT


