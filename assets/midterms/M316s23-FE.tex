% !TEX TS-program = pdflatexmk
\documentclass[11pt]{article}
\usepackage[margin=.8in]{geometry}
\usepackage{amsmath,amssymb,amsthm, latexsym, mathrsfs, pdfsync, multicol,
fancybox, fancyhdr,
graphicx, enumerate,
subfig, tikz, pgfplots,array}

%\singlespacing
\def\RR{{\mathbb R}}
\def\NN{{\mathbb N}}
\def\ZZ{{\mathbb Z}}
\def\QQ{{\mathbb Q}}
\def\CC{{\mathbb C}}
\def\bc{\begin{center}}
\def\ec{\end{center}}
\def\be{\begin{enumerate}}
\def\ee{\end{enumerate}}
\def\bi{\begin{itemize}}
\def\ei{\end{itemize}}
\def\t{\times}
\newcommand{\ol}[1]{\overline{#1}}
\newcommand{\oimp}[1]{\overset{#1}{\Longleftrightarrow}}
\newcommand{\bv}[1]{\ensuremath{ \mathbf{\vec{#1}}} }
\renewcommand{\d}{\displaystyle}
\newcommand{\blank}[1]{\rule{#1}{0.75pt}}

\usetikzlibrary{calc}

\lhead{\sc{Math 316 Hist. of Math.}}
\chead{\large \sc Final Exam} 
\rhead{\sc Spring 2023}
\cfoot{}
\pagestyle{fancy}
%
\begin{document}
\thispagestyle{fancy}

\vspace{1in}

\textbf{Student Name:}\hspace{2in}

\vspace{1in}

{
\renewcommand{\baselinestretch}{1.8}
\setlength{\tabcolsep}{.2in}
\normalsize
\begin{center}
\begin{tabular}{|c|c|c|c|}
\hline
&Problem&Total Points&\parbox{.8in}{\hfil Score\hfil}\\
\hline
\hline
&1&15&\\
\hline
&2&15&\\
\hline
&3&15&\\
\hline
&4&10&\\
\hline
&5&15&\\
\hline
&6&30&\\
\hline
\hline
Total&100&&\\
\hline
\end{tabular}

\end{center}
}

\vspace{1in} 

Guidelines
\begin{itemize}
\item You have 2 hours to take the exam.
\item You may use a calculator and two pages of notes (i.e. two sheets of paper with writing on both sides of each sheet).
\end{itemize}

\newpage


\vspace*{-0.3in}


\be
%%%False position
\item (15 points)
	\be
	\item Use the \emph{method of false position} to solve Problem 31 of the Rhind Papyrus stated below:
	
	\begin{quote} A quantity, and its $\frac{2}{3}$, its $\frac{1}{2}$, and its $\frac{1}{7}$ add together become 33. What is the quantity?
	\end{quote}
	
	\vfill
	\item The method of false position was a common strategy for solving equations though the Middle Ages in Europe. What made this method so durable?
	\vfill
	\item What are some of the limitations of this method and why is it no longer common to find the method taught in school?
	\vfill
	\ee
\newpage

\item (15 points)
	\begin{enumerate}
	\item Use the ancient Egyptian method to find 163 divided by 30.
	\vfill
	\item Demonstrate how ancient Greek mathematicians could \emph{construct} a line segment of length $\sqrt{2}$ even if they didn't feel $\sqrt{2}$ could be a number.
	
	\vfill 
	\item A modern human might answer that $163 \div 30=5.433$ or describe $\sqrt{2}$ as $1.4142.$ Give some examples of how how our modern notion of number and notation for number has evolved since the ancient Greeks and Egyptians.
	\vfill
	\end{enumerate}
\newpage
\item (15 points) The following questions concern Euclid's \emph{Elements of Geometry}.
	\begin{enumerate}
	\item Why did the fifth postulate of Book I receive so much attention by so many mathematicians? (Your answer should be limited to the motivation of mathematicians in roughly the first 1000 years after the \emph{Elements} was written.)
	\vfill
	\item With what two propositions does Book I end?
	\vspace{1in}
	\item Describe at least three \textbf{distinct} ways in which Euclid's \emph{Elements} influenced later mathematics.
	\vfill	
	\end{enumerate}
\newpage
\item (10 points)
	\be
	\item Solve Problem 27 from Book I of Diophantus' \emph{Arithmetica}:
	\begin{quote} Find two numbers such that their sum and product are given numbers: say their sum is 20 and their product is 96.
	\end{quote}
	Hint: Call the numbers $10+x$ and $10-x$.
	\vfill
	\item Explain what is meant today by a Diophantine equation and how this differs from what Diophantus meant in his \emph{Arithmetica}.
	\vfill
	\ee
	\newpage
\item (15 points)
	\be
	\item Explain how the algebra of Muhammad al-Khwarizmi differed from the algebra of Francois Vieta.
	\vfill
	\item Explain why algebraic notion was important in the development of analytic geometry.
	\vfill
	\item Explain why the development of analytic geometry was important to the development of calculus.
	\vfill
	\ee
\newpage
\item (30 points) Short Answer.
	\be
	\item Give two examples illustrating that \emph{how} humans write mathematics influences how we \emph{do} mathematics. One example should predate 1000 AD and one should come after 1000 AD.
	\vfill 
	\item Describe the origins of the mathematical subject we now call \emph{Trigonometry}. Who is responsible for the unit circle definition of trigonometric functions taught in modern classrooms?
	\vfill
	\item Describe the origins of the mathematical subject we now call \emph{Probability}.
	\vfill
	\newpage
	\item State two mathematical accomplishments of Archimedes.
	\vfill
	\item Describe one of the first appearances of negative numbers and what motivates them. Include names and dates.
	\vfill
	\item Describe one of the first appearances of imaginary numbers and what motivates them. Include names and dates.
	\vfill
	\ee
\ee
\end{document}
%%%%%%%%%%%%%%%%%%%%%%%%%%
%%%%%%%END
%%%%%%%%%%%%%%%%%%%%%%%%%%


 