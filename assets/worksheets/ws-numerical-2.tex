% !TEX TS-program = pdflatexmk
\documentclass[12pt]{article}

% Layout.
\usepackage[top=1in, bottom=0.75in, left=1in, right=1in, headheight=1in, headsep=6pt]{geometry}

% Fonts.
\usepackage{mathptmx}
\usepackage[scaled=0.86]{helvet}
\renewcommand{\emph}[1]{\textsf{\textbf{#1}}}

% Misc packages.
\usepackage{amsmath,amssymb,latexsym}
\usepackage{graphicx,hyperref}
\usepackage{array}
\usepackage{xcolor}
\usepackage{multicol}
\usepackage{tabularx,colortbl}
\usepackage{enumitem}

\usepackage{fancyhdr}
\pagestyle{fancy} 
\lhead{\large\sf\textbf{MATH 316: History of Math}}
\rhead{\large\sf\textbf{Worksheet (\S 1.3)}}

\begin{document}
\begin{enumerate}
\item Translate the (cartoon of) Babylonian cuneiform script into our numerals
	\vfill
\item Each number below is written \emph{base 60}, where the semicolon is used in place of a sexagesimal point. Rewrite them as a base 10 number using a decimal point!
	\begin{enumerate}
	\item $5,46,10$\\
	\vfill
	\item $1,0;28,59$\\
	\vfill
	\end{enumerate}
	
\item Write each decimal number below in base 60.
	\begin{enumerate}
	\item $62.3$\\
	\vfill
	\item $62.33333...$\\
	\vfill
	\item $300,000$
	\vfill
	\end{enumerate}
\end{enumerate}
\end{document}