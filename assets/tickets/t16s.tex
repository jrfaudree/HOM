% !TEX TS-program = pdflatexmk
\documentclass[12pt]{article}

% Layout.
\usepackage[top=1in, bottom=0.75in, left=1in, right=1in, headheight=1in, headsep=6pt]{geometry}

% Fonts.
\usepackage{mathptmx}
\usepackage[scaled=0.86]{helvet}
\renewcommand{\emph}[1]{\textsf{\textbf{#1}}}

% Misc packages.
\usepackage{amsmath,amssymb,latexsym}
\usepackage{graphicx,hyperref}
\usepackage{array}
\usepackage{xcolor}
\usepackage{multicol}
\usepackage{tabularx,colortbl}
\usepackage{enumitem}

\usepackage{fancyhdr}
\pagestyle{fancy} 
\lhead{\large\sf\textbf{MATH 316: History of Math}}
\rhead{\large\sf\textbf{Ticket 16: Section 5.3 and 5.5}}

\begin{document}
Jill's Solutions \\


 For each mathematician below, provide several biographical details (where, when at least) and at least two mathematical accomplishments. (The first 6 appear in Section 5.3, the rest in Section 5.5.)
 \begin{enumerate}
 \item Aryabhata (476-550 AD) India\\
 Wrote \textit{Aryabhatiya}, astronomical work, calculated $\pi$, summation of arithmetic and geometric series, made tables of sine values, know how to solve linear indeterminant equations in two variables.
 \vfill
 \item Brahmagupta (around 625 AD) India\\
 Introduced negative numbers including rules about how to operate with them and allowing them as solutions to equations. Produced complete solution to linear indeterminant equations in two variables. He also considered quadratic indeterminant equations.
 \vfill
 \item Mahavira (around 850 AD) India\\
Updated the work of Brahmagupta and is one of the ways Brahmagupta's work has been preserved. Used base-10, place-value numerical system with 10 symbols. That is, he had a representative for zero and even stated that zero times zero is zero. He described an algorithm for calculating the number of combinations of $n$ things taken $r$ at a time.
 \vfill
 \item Bhaskara (1114-1185 AD) India\\
 Our text mentions three different texts about astronomy, algebra and arithmetic. This includes solutions to (fanciful) linear indeterminant equations. He also made tables of trigonometric values.
 \vfill
 \item Chang Ch'iu-chien (around 490 AD) China\\
 Also written Zhang Quijian or Chang Ch'iu-Chin. Wrote \textit{The Mathematical Classic of Zhang Quijian}. It included the 100 fowls problem. It contains other indeterminant problems.
 \vfill
 \item Sun-Tsu or Sun-Tzu or Sun Zi (250 AD) China\\
 Solved problems that would be described as systems of linear congruences and other diophantine equations. His work includes the Chinese Remainder Theorem.
 \vfill
 \item Muhammad Al-Khwarizmi (780-850 AD) Baghdad in present day Iraq\\
 The word \textit{algorithm} is derived from his name. Our word \textit{algebra} is the result of European translation of the ``al-jabr" from his work \textit{Hisab al-jabr w'al-muqabala}. His work is the first known Arabic work that included the Hindu base-10, positional numerical system. It does not appear that zero held a position equal to the other 9 symbols. He demonstrated solutions to all types of quadratic equations and gave proofs of their correctness.
 \vfill
 \item Abu Kamil (850-930 AD) of Egyptian descent\\
 A commentator on Al-Khowarizmi. He added his own problems and his own, novel, solutions. Developed an algebra of radicals. His work had a profound impact on how computation and algebra reached Europe via Fibonacci.
 \vfill
 \item Thabit ibn Qurra (836-901 AD) \\
 physician and translator. Translated Apollonius' \textit{Conic Sections}, Nicomachus' \textit{Introduction Arithmeticae}, and Archimedes' \textit{On the Measurement of the Circle} and \textit{On the Sphere and the Cylinder}. Did work an amicable numbers. Added to the work on number theory. Gave a generalization of the Pythagorean Theorem.\\
 
 \item Omar Khayyam (1048-1123 AD) Persia\\
 Constructs solutions to all kinds of cubic equations using intersection conic sections. Also, authored a commentary on Euclid's \textit{Elements} with a focus on the fifth axiom and parallel lines.
 \vfill
 \end{enumerate}

\end{document}