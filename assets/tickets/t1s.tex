% !TEX TS-program = pdflatexmk
\documentclass[12pt]{article}

% Layout.
\usepackage[top=1in, bottom=0.75in, left=1in, right=1in, headheight=1in, headsep=6pt]{geometry}

% Fonts.
\usepackage{mathptmx}
\usepackage[scaled=0.86]{helvet}
\renewcommand{\emph}[1]{\textsf{\textbf{#1}}}

% Misc packages.
\usepackage{amsmath,amssymb,latexsym}
\usepackage{graphicx,hyperref}
\usepackage{array}
\usepackage{xcolor}
\usepackage{multicol}
\usepackage{tabularx,colortbl}
\usepackage{enumitem}

\usepackage{fancyhdr}
\pagestyle{fancy} 
\lhead{\large\sf\textbf{MATH 316: History of Math}}
\rhead{\large\sf\textbf{Ticket 1 (\S 1.1-1.2)}}

\begin{document}

\begin{center}{Jill's Solutions}\end{center}
\begin{enumerate}

\item Read Burton \S 1.1-1.2. Summarize the mathematical topic discussed in these sections using at most two sentences.\\

This section is about early forms of numerical representation. Most of the examples are not positional and representation is limited to that of whole numbers: 1,2,3,... and maybe zero.\\

\item For each item/numerical system mentioned in this section, state the rough time period, an approximate location, and a brief description of the numerical system.
	\begin{enumerate}
	\item Ishango Bone (pictured on page   5)
		\begin{enumerate}
		\item time period: 20,000 years ago
		\item location: central Africa
		\item description: numbers represented as tally marks, material: bone.
		\vfill
		\end{enumerate}
	\item \textbf{Inca} quipus 
		\begin{enumerate}
		\item time period: 1200-1533 AD
		\item location: Peru
		\item description: powers of 10 represented by knots in string. positional. implied zero but no symbol for it. How many different \emph{types} of knots? 2 or 9 depending on your definition ``different." material: cord.
		\vfill
		\end{enumerate}
	\item Mayan symbols
		\begin{enumerate}
		\item time period: 300-900 AD
		\item location: Mexico and Central America
		\item description: positional base 20-ish, three symbols (dot, line, shell) to  represent 1, 5 and zero. material: stone
		\vfill
		\end{enumerate}
	\item Egyptian Heiroglyphs (pages 13-14)
		\begin{enumerate}
		\item time period: as early as 3500 BC a base 10 symbolic system allowing for the representation of arbitrarily large numbers
		\item location: Egypt
		\item description: base 10 symbolic (not positional), a new symbol for each power of 10. material: stone. Note, the oldest Egyptian papyrus is estimated to be from 2560-2550 BC found in 2013.
		\vfill
		\end{enumerate}
	\item Hieratic Symbols (pages 15-16)
		\begin{enumerate}
		\item time period: developed 3000-2000 BC
		\item location: Egypt
		\item description: a cursive version of the hieroglyphs for the purpose of writing on papyri. The number of \textbf{distinct} symbols increased. For example, in hieroglyphs, the integers 1 to 9 were represented by tally marks where hieratic used 9 different symbols. Similar expansion happened for the representation of 10 - 90. material: papyrus.
		\vfill
		\end{enumerate}
	\item Alphabetic System (page 16-17)
		\begin{enumerate}
		\item time period: 500-400 BC
		\item location: ancient Ionia (in present day Turkey) Why is it called a \textbf{Greek} system? The language/writing is in ancient Greek.
		\item description: base 10. symbolic, not positional. Individual letters to represent numbers 1-9, 10,20,..,90, 100, 200,...900. (Explains numerology!) material: papyrus
		\vfill
		\end{enumerate}
	\end{enumerate}
\end{enumerate}

\end{document}