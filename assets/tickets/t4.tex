% !TEX TS-program = pdflatexmk
\documentclass[12pt]{article}

% Layout.
\usepackage[top=1in, bottom=0.75in, left=1in, right=1in, headheight=1in, headsep=6pt]{geometry}

% Fonts.
\usepackage{mathptmx}
\usepackage[scaled=0.86]{helvet}
\renewcommand{\emph}[1]{\textsf{\textbf{#1}}}

% Misc packages.
\usepackage{amsmath,amssymb,latexsym}
\usepackage{graphicx,hyperref}
\usepackage{array}
\usepackage{xcolor}
\usepackage{multicol}
\usepackage{tabularx,colortbl}
\usepackage{enumitem}

\usepackage{fancyhdr}
\pagestyle{fancy} 
\lhead{\large\sf\textbf{MATH 316: History of Math}}
\rhead{\large\sf\textbf{Ticket 4 (\S 2.4-2.6)}}

\begin{document}


\begin{enumerate}
\item Read Burton \S 2.4-2.6. Summarize the mathematical topics discussed in these sections using at most two sentences.
\vspace{1in}
\item Roughly when was the Moscow papyrus written?\\

\item On page 54 is an ancient Egyptian formula for the area of a quadrilateral: $A=\frac{1}{4}(a+c)(b+d).$
	\begin{enumerate}
	\item Use it to find the area of a rectangle with side lengths $a$ and $b$.
	\vfill
	\item Use it to find the area of the quadrilateral below and compare that calculation to a correct calculation. Does the Egyptian formula give an over-estimate or an underestimate of the exact area?\\
	\vspace{1.5in}
	\item Give a simple argument or proof-by-picture to demonstrate that the formula the Egyptian were using is not only incorrect, but obviously so. In particular, that official using the formula must have known it was incorrect. 
	\vfill
	\end{enumerate}
\item Write a sentence or two explaining \textbf{to a 5th grader} what $\pi$ is.
\vfill
\item How much of your answer is something an ancient Egyptian could understand?
\vfill
\end{enumerate}

\end{document}