% !TEX TS-program = pdflatexmk
\documentclass[12pt]{article}

% Layout.
\usepackage[top=1in, bottom=0.75in, left=1in, right=1in, headheight=1in, headsep=6pt]{geometry}

% Fonts.
\usepackage{mathptmx}
\usepackage[scaled=0.86]{helvet}
\renewcommand{\emph}[1]{\textsf{\textbf{#1}}}

% Misc packages.
\usepackage{amsmath,amssymb,latexsym}
\usepackage{graphicx,hyperref}
\usepackage{array}
\usepackage{xcolor}
\usepackage{multicol}
\usepackage{tabularx,colortbl}
\usepackage{enumitem}

\usepackage{fancyhdr}
\pagestyle{fancy} 
\lhead{\large\sf\textbf{MATH 316: History of Math}}
\rhead{\large\sf\textbf{Ticket 12 Sections 4.4 and 4.5}}

\begin{document}

Jill's Solutions

\begin{enumerate}
\item Provide some basic \textbf{biographical information} about Eratosthenes including approximate dates, locations, travels, etc.\\

(276-194 BC) so after Euclid but contemporary with Archimedes and Apollonius. Born in Cyrene. Lived mostly in Alexandria eventually being head librarian. Studied at Plato's school in Athens.
\vfill
\item What is the Sieve of Eratosthenes?\\

A method of identifying primes by iteratively deleting multiples of primes.\\
\vfill
\item Provide some basic \textbf{biographical information} about Claudius Ptolemy, the author of \textit{Syntaxis Mathematica} including approximate dates, locations, etc.\\

(100-170 AD) so after Archimedes and Apollonius. Native of Egypt. Probably at the Museum in Alexandria.
\vfill
\item Provide a few facts about the text \textit{Syntaxis Mathematica} such as its topic, its assumptions, and/or its influence.\\

Better known as Almagest (the Greatest). A summary of what was known about astronomy. Assumed an earth-centered universt.
\vfill
\item Provide some basic \textbf{biographical information} about Archimedes.\\

(287-212BC) So a bit after Euclid and contemporary with Apollonius and Eratosthenes. Lived in Syracuse, Sicily.
\vfill
\item List three manuscripts authored by Archimedes and roughly what they contained.\\

On the Sphere and the Cylinder. It contains formulas-ish relating surface area and volumes of spheres and cylinders.\\
The Measurement of the Circle. Formulas-ish relating circumference and area of a circle. It includes an estimate of $\pi$ and a clear algorithm for improving the estimation.\\
The Sand Reckoner. Invents mechanisms for expressing very large numbers.\\
On Spirals. Quadrature of a Parabola.

\vfill
\item Provide some basic \textbf{biographical information} about Apollonius of Perga.\\

(262-190 BC)So a bit after Euclid and contemporary with Archimedes and Eratosthenes
\vfill
\end{enumerate}
\end{document}