% !TEX TS-program = pdflatexmk
\documentclass[12pt]{article}

% Layout.
\usepackage[top=1in, bottom=0.75in, left=1in, right=1in, headheight=1in, headsep=6pt]{geometry}

% Fonts.
\usepackage{mathptmx}
\usepackage[scaled=0.86]{helvet}
\renewcommand{\emph}[1]{\textsf{\textbf{#1}}}

% Misc packages.
\usepackage{amsmath,amssymb,latexsym}
\usepackage{graphicx,hyperref}
\usepackage{array}
\usepackage{xcolor}
\usepackage{multicol}
\usepackage{tabularx,colortbl}
\usepackage{enumitem}

\usepackage{fancyhdr}
\pagestyle{fancy} 
\lhead{\large\sf\textbf{MATH 316: History of Math}}
\rhead{\large\sf\textbf{Ticket 2 (\S 1.3)}}

\begin{document}

\begin{center}{Jill's Solutions}\end{center}
\begin{enumerate}
\item Read Burton \S 3. Summarize the mathematical topic discussed in these sections using at most two sentences.\\

This sections continues the discussion of numerical representation with a focus on that of ancient Babylonians and a brief discussion of Chinese and Indian versions.\\

\item Add details of Babylonian cuneiform script as on the previous ticket.
	\begin{enumerate}
	\item time period: around 2000 BC
	\item location: Mesopotamia (present day Iraq)
	\item description: two symbols used in a symbolic way (1 and 10) inside a positional base 60 system, later (around 300 BC) used a divider or space holder to indicate a missing position but not really considered a zero. material: clay
	\end{enumerate}
\vfill
	
\item Add details of the Chinese numerical system of around 1400 BC (see the bottom of page 26)
	\begin{enumerate}
	\item time period: around 1400 BC
	\item location: China (Ming Dynasty)
	\item description: positional with 9 symbols (ie no zero)
	\end{enumerate}
\vfill
	
\item What is the name and date of the oldest known text from India or China?
	\begin{enumerate}
	\item time: 150 BC
	\item location: China (Han Dynasty)
	\item name of text: Nine Chapters on the Mathematical Arts
	\item name of author: unknown. We know of this work from \emph{commentary} on the work by a Chinese mathematician named Liu Hui around 260 AD. 
	\end{enumerate}
		
\item So far, the text has devoted 2 full pages to ancient Chinese and Indian numerical representation compared to more than 10 full pages devoted to numerical systems of the ancient people in the Mediterranean (Greeks and Egyptians) and Babylonians (just east of these). What is the author's explanation of this?\\

Due to a wetter climate and less durable materials (bamboo or silk versus clay), fewer Chinese texts have survived. Thus, there is much less information about what Chinese mathematicians were thinking and when they were thinking it. \\

The author does not say this explicitly but proximity and cultural bias surely plays a role. European countries invaded and ransacked Egypt and surrounds, bring back artifacts to be studied. Indeed, Western Europe has had a kind of romanticization/obsession with ancient Egypt for a long time (think Napoleon's campaign, the familiarity of the Rosetta Stone, or the giant Pyramid built in Memphis (now a Bass Pro store). 

\end{enumerate}

\end{document}