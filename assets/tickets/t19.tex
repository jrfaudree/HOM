% !TEX TS-program = pdflatexmk
\documentclass[12pt]{article}

% Layout.
\usepackage[top=1in, bottom=0.75in, left=1in, right=1in, headheight=1in, headsep=6pt]{geometry}

% Fonts.
\usepackage{mathptmx}
\usepackage[scaled=0.86]{helvet}
\renewcommand{\emph}[1]{\textsf{\textbf{#1}}}

% Misc packages.
\usepackage{amsmath,amssymb,latexsym}
\usepackage{graphicx,hyperref}
\usepackage{array}
\usepackage{xcolor}
\usepackage{multicol}
\usepackage{tabularx,colortbl}
\usepackage{enumitem}

\usepackage{fancyhdr}
\pagestyle{fancy} 
\lhead{\large\sf\textbf{MATH 316: History of Math}}
\rhead{\large\sf\textbf{Ticket 19: Section 8.1}}

\begin{document}
\begin{enumerate}
\item Read Section 8.1 and answer the questions below.
\item Describe Galileo Galilei's \textit{Dialogue Concerning the Two Chief World Systems}. (Hereafter called \textit{Dialogue}. When was it published? What was it about? What was its impact?
\vfill
\item Return to the bottom of page 345 - 346. List three ``firsts" in modern algebraic notation. State the notation, the name of the inventor, the text in which it appeared and the date. For example, in 1557, Robert Recorde published \textit{The Whetstone of Witte} introducing the symbol $=$ (thought it was somewhat longer!).
\vfill
\item Return to the bottom of page 347 and describe the advance in algebraic notation attributed to Francois Vieta and Rene Descartes. Include approximate dates.
\vfill
\item What was the substance of Simon Stevins \textit{The Tenth} and approximately when was it published?
\vfill
\item Provided approximate dates for Nicolas Copernicus, Johannes Kepler, and Tycho Brache and a short summary of their contributions to astronomy.
\vfill
\end{enumerate}
\end{document}