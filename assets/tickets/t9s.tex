% !TEX TS-program = pdflatexmk
\documentclass[12pt]{article}

% Layout.
\usepackage[top=1in, bottom=0.75in, left=1in, right=1in, headheight=1in, headsep=6pt]{geometry}

% Fonts.
\usepackage{mathptmx}
\usepackage[scaled=0.86]{helvet}
\renewcommand{\emph}[1]{\textsf{\textbf{#1}}}

% Misc packages.
\usepackage{amsmath,amssymb,latexsym}
\usepackage{graphicx,hyperref}
\usepackage{array}
\usepackage{xcolor}
\usepackage{multicol}
\usepackage{tabularx,colortbl}
\usepackage{enumitem}

\usepackage{fancyhdr}
\pagestyle{fancy} 
\lhead{\large\sf\textbf{MATH 316: History of Math}}
\rhead{\large\sf\textbf{Ticket 9 Sections 4.1-4.2}}

\begin{document}


\begin{enumerate}
\item Read Section 4.1. Summarize in 1-2 sentences.

This section describes the origins of Alexandria, the Museum, its library and the historical prominence of Euclid and his text \emph{Elements}. The Museum feels akin to a very cushy modern research university. (Some additional context: Euclid's work is dated roughly 300 BC. This Chapter will cover roughly 300 BC to 27 BC which is roughly the time time from the rise of Alexander the Great to the rise of the Roman Empire.)

\item Read Section 4.2. (lightly). Summarize in 1-2 sentences.

This section discusses Euclid's text the \emph{Elements} including its contents and its influence. It is and has been a model of mathematical exposition with a focus on logical structure and efficiency over intuition or motivation. The intellectual exercise of mastering this approach to geometry as long been viewed as a means of training the mind and thus has traditionally been a mandatory part of a liberal education. It is also an example of a text that effectively eliminated the presumably extensive works that preceded it and one for which there are no original copies. It is supposed to be a summative text, not one of recent research.

\item Review the 5 postulates on page 146. Explain what is peculiar about the 5th one. Draw it for yourself.

Its statement is at least three times longer than the other four postulates and it really requires a picture to understand. The others are so obvious no picture is necessary.

\vfill
\item For each common notion, write its algebraic equivalent.
\begin{enumerate}
\item If $a=b$ and $b=c$, then $a=c.$
\item If $a=b$ and $c=d$, then $a+c=b+d.$
\item If $a=b$ and $c=d$, then $a-c=b-d.$
\item Uuuhh... this isn't algebraic. This is about moving one shape on top of another.
\item If $a+b=c,$ then $a<c$ and $b<c.$ (Which is only true if $a$, $b$, and $c$ are all positive!
\end{enumerate}

\item Read the statement and proof of Proposition 16 on page 152. What Common Notions, Postulates, and Propositions does the proof use?

Postulates 1 and 2 are used to draw various lines and extensions of lines. Proposition 4 is used to show various triangles are congruent. It also uses the fact that line segments can be bisected (Prop 10). Proposition 15 about vertical angles, Common notion 5 about the whole being larger than the part, and Postulate 2 about extending a line. Implicitly it uses that idea that, in the Euclidean plane, the ray defined by BF will always be the same side of the line determined by BD.

\end{enumerate}
\end{document}