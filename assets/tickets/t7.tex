% !TEX TS-program = pdflatexmk
\documentclass[12pt]{article}

% Layout.
\usepackage[top=1in, bottom=0.75in, left=1in, right=1in, headheight=1in, headsep=6pt]{geometry}

% Fonts.
\usepackage{mathptmx}
\usepackage[scaled=0.86]{helvet}
\renewcommand{\emph}[1]{\textsf{\textbf{#1}}}

% Misc packages.
\usepackage{amsmath,amssymb,latexsym}
\usepackage{graphicx,hyperref}
\usepackage{array}
\usepackage{xcolor}
\usepackage{multicol}
\usepackage{tabularx,colortbl}
\usepackage{enumitem}

\usepackage{fancyhdr}
\pagestyle{fancy} 
\lhead{\large\sf\textbf{MATH 316: History of Math}}
\rhead{\large\sf\textbf{Ticket 7 (\S 3.3-3.5)}}

\begin{document}


\begin{enumerate}
\item Read Burton \S 3.3-3.5 and you are given permission to do so \emph{lightly}. 

In section 3.3, our focus will be on (i) geometric proofs of the Pythagorean Theorem, (ii) the word \emph{incommensurable}, and (iii) what is meant by the \emph{crisis of incommensurable quantities}.

In Section 3.4, our focus is on (i) what counts as a \emph{geometric construction}, (ii) what are the three construction problems of antiquity, which requires knowing the word \emph{quadrature}. In class, we will show Hippocrates' quadrature of the lune.

In Section 3.5, our focus is on the curve called a \emph{quadratrix}, its definition and its uses.

\begin{center} Section 3.3 Questions \end{center}

\item Supposing that the triangle with sides $a$, $b$ and $c$ is a right triangle, explain how the pictures below ``prove" that $a^2+b^2=c^2.$ What facts from geometry are being \emph{implicitly} used in these pictures/arguments?

\includegraphics[scale=.7]{t7pic}\\
\vfill
\item (pg 109) What does it mean for two line segments to be \emph{commensurable}? What does it mean for two line segments to be \emph{incommensurable}? (Note that in neither case are you allowed to use the words rational or irrational.)\\
\vfill

\item In one sentence, what was the \emph{crisis of incommensurable quantities}? \\
\vfill

\item Using modern terminology and in one sentence, why is this a \emph{crisis}?\\
\vfill

\newpage
\begin{center} Section 3.4 Questions \end{center}

\item What do ancient Greek mathematicians mean by a \emph{geometric construction}? (pg 121)\\
\vfill

\item Restate in modern language the \emph{quadrature of the circle} problem.\\
\vfill

\item What does the author mean by the \emph{three construction problems of antiquity}? Why does he devote an entire section to them?\\
\vfill


\begin{center} Section 3.5 Questions \end{center}

\item How is the curve called a \emph{quadratrix} defined?\\
\vfill

\item Why is the curve important?\\
\vfill


\end{enumerate}
\end{document}