% !TEX TS-program = pdflatexmk
\documentclass[12pt]{article}

% Layout.
\usepackage[top=1in, bottom=0.75in, left=1in, right=1in, headheight=1in, headsep=6pt]{geometry}

% Fonts.
\usepackage{mathptmx}
\usepackage[scaled=0.86]{helvet}
\renewcommand{\emph}[1]{\textsf{\textbf{#1}}}

% Misc packages.
\usepackage{amsmath,amssymb,latexsym}
\usepackage{graphicx,hyperref}
\usepackage{array}
\usepackage{xcolor}
\usepackage{multicol}
\usepackage{tabularx,colortbl}
\usepackage{enumitem}

\usepackage{fancyhdr}
\pagestyle{fancy} 
\lhead{\large\sf\textbf{MATH 316: History of Math}}
\rhead{\large\sf\textbf{Ticket 10 Section 4.3}}

\begin{document}

\begin{center} Jill's Solutions \end{center}

\begin{enumerate}
\item Read Section 4.3. Summarize in 1-2 sentences.\\

This section describes some of the number theory in Euclid's Elements.
\vfill
\item Write the prime factorizations of $6120$ and $14850.$\\

$6120=2^3\cdot3^2\cdot 5 \cdot 17$ and $14,850=2 \cdot 3^3 \cdot 5^2\cdot 11$\\
\vfill
\item Use the prime factorization you found above to determine the greatest common divisor of $6120$ and $14850.$ (i.e. $\text{gcd}(6120,14850).$\\

$\text{gcd}(6120,14850)=2 \cdot 3^2 \cdot 5=90.$
\vfill
\item Use the Euclidean Algorithm (description page 174, example page 175) to find the greatest common divisor of $6120$ and $14850.$ \\

\begin{tabular}{rcl}
$14850$&$=$&$2(6120)+2610$\\
$6120$&$=$&$2(2610)+900$\\
$2610$&$=$&$2(900)+810$\\
$900$&$=$&$810+90$\\
$810$&$=$&$9(90)$\\
\end{tabular}
\vfill
\item Give examples of positive integers $a$, $b$, and $c$ such that $a$ divides $c$ and $b$ divides $c$ but $ab$ does not divide $c$.\\

Choose $a=4$, $b=6$, and $c=12.$
\vfill
\item Find $a$ and $b$ so that $19a+13b=1.$\\

$1=3 \cdot 13 - 2 \cdot 19$
\vfill
\end{enumerate}
\end{document}