% !TEX TS-program = pdflatexmk
\documentclass[12pt]{article}

% Layout.
\usepackage[top=1in, bottom=0.75in, left=1in, right=1in, headheight=1in, headsep=6pt]{geometry}

% Fonts.
\usepackage{mathptmx}
\usepackage[scaled=0.86]{helvet}
\renewcommand{\emph}[1]{\textsf{\textbf{#1}}}

% Misc packages.
\usepackage{amsmath,amssymb,latexsym}
\usepackage{graphicx,hyperref}
\usepackage{array}
\usepackage{xcolor}
\usepackage{multicol}
\usepackage{tabularx,colortbl}
\usepackage{enumitem}

\usepackage{fancyhdr}
\pagestyle{fancy} 
\lhead{\large\sf\textbf{MATH 316: History of Math}}
\rhead{\large\sf\textbf{Ticket 2 (\S 1.3)}}

\begin{document}


\begin{enumerate}
\item Read Burton \S 3. Summarize the mathematical topic discussed in these sections using at most two sentences.
\vspace{1in}
\item Add details of Babylonian cuneiform script as on the previous ticket.
	\begin{enumerate}
	\item time period:
	\item location:
	\item description: 
	\end{enumerate}
\vfill
	
\item Add details of the Chinese numerical system of around 1400 BC (see the bottom of page 26)
	\begin{enumerate}
	\item time period:
	\item location:
	\item description: 
	\end{enumerate}
\vfill
	
\item What is the name and date of the oldest known text from India or China?
	\begin{enumerate}
	\item time:
	\item location:
	\item name of text: 
	\item name of author:
	\end{enumerate}
		
\item So far, the text has devoted 2 full pages to ancient Chinese and Indian numerical representation compared to more than 10 full pages devoted to numerical systems of the ancient people in the Mediterranean (Greeks and Egyptians) and Babylonians (just east of these). What is the author's explanation of this?
\vfill

\end{enumerate}

\end{document}