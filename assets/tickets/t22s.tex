% !TEX TS-program = pdflatexmk
\documentclass[12pt]{article}

% Layout.
\usepackage[top=1in, bottom=0.75in, left=1in, right=1in, headheight=1in, headsep=6pt]{geometry}

% Fonts.
\usepackage{mathptmx}
\usepackage[scaled=0.86]{helvet}
\renewcommand{\emph}[1]{\textsf{\textbf{#1}}}

% Misc packages.
\usepackage{amsmath,amssymb,latexsym}
\usepackage{graphicx,hyperref}
\usepackage{array}
\usepackage{xcolor}
\usepackage{multicol}
\usepackage{tabularx,colortbl}
\usepackage{enumitem}

\usepackage{fancyhdr}
\pagestyle{fancy} 
\lhead{\large\sf\textbf{MATH 316: History of Math}}
\rhead{\large\sf\textbf{Ticket 22: Section 8.4}}

\begin{document}
\begin{enumerate}
\item Read Section 8.4.
\item This section discusses the dispute of priority of calculus. This question asks you to argue on both sides.
	\begin{enumerate}
	\item Give a brief argument in favor of assigning priority to Newton.\\
	
	Both Newton and Leibniz used series in their work and it is does appear that Leibniz was made aware of Newton's work on this topic via Oldenburg. If one takes the two mathematicians at their word, it does appear that the foundational ideas were understood by Newton in the 1660's prior to Leibniz working them out in the 1670s. Leibniz did not produce a masterful application of the calculus the way Newton did in his Principia.
	\vfill
	\item Give a brief argument explaining why Newton and Leibniz should share the title of Inventor of Calculus.\\
	
	It is clear that, whatever Leibniz got from Newton, Leibniz formulation of the calculus and the fundamental theorem of calculus was different. While Newton did demonstrate the power of calculus to understand the world, Leibniz gave the notation and framework that illuminated that FTC and unequivocally influenced how we write and think about the topic.
	\vfill
	\end{enumerate}
\item What is your opinion regarding this dispute? 
\vfill
\item This section ends with a mention of two women. Why are they mentioned here? Provide a couple of illuminating facts about them.\\

Earlier, we learned the value of Van Schooten's translation and elaboration of Descartes work, these two women similarly produced translation and elaboration of the ideas of calculus in the work of Newton and Leibniz.
	\begin{enumerate}
	\item Maria Agnesi (1718-1799)\\
	
	Itialian. The child of a math professor (not unlike Hypatia!) Published a work called \textit{Instituzioni Analitiche} 1748 that grew out of educating her younger brothers. Was considered model of clarity and exposition unifying algebra and calculus. Intended for learners not existing scholars.
	\vfill
	\item Emilie du Ch\^atelet (1701-1749)\\
	French. Aristocrat. Pursued education in mathematics via private tutoring. Mistress of Voltaire. Made a French translation and elaboration of Newton's Principia. Published posthumously in1759. 
	\vfill
	\end{enumerate}

\end{enumerate}
\end{document}