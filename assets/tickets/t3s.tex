% !TEX TS-program = pdflatexmk
\documentclass[12pt]{article}

% Layout.
\usepackage[top=1in, bottom=0.75in, left=1in, right=1in, headheight=1in, headsep=6pt]{geometry}

% Fonts.
\usepackage{mathptmx}
\usepackage[scaled=0.86]{helvet}
\renewcommand{\emph}[1]{\textsf{\textbf{#1}}}

% Misc packages.
\usepackage{amsmath,amssymb,latexsym}
\usepackage{graphicx,hyperref}
\usepackage{array}
\usepackage{xcolor}
\usepackage{multicol}
\usepackage{tabularx,colortbl}
\usepackage{enumitem}

\usepackage{fancyhdr}
\pagestyle{fancy} 
\lhead{\large\sf\textbf{MATH 316: History of Math}}
\chead{\sf\textbf{Jill's Solutions}}
\rhead{\large\sf\textbf{Ticket 3 (\S 2.1-2.2)}}

\begin{document}


\begin{enumerate}
\item Read Burton \S 2.1-2.2. Summarize the mathematical topic discussed in these sections using at most two sentences.
\vspace{1in}
\item Describe the \emph{Rhind papyrus}
	\begin{enumerate}
	\item written roughly when: 1650 BC
	\item author:
	 scribe Ahmes
	\item language/script: 
	 Egyptian heiratic
	\item dimensions: 
	18 feet long, 13 inches tall
	\item when and where was it purchased:
	1858 in Luxor Eqypt by Scotsman Henry Rhind. Now in British Museum.
	\item what sort of math is in it?
	 practical problems (85 of them) providing instruction on multiplication, division, fractions and geometry. Ahmes characterizes the work as reproducing earlier, known materials.
	\vfill
	\end{enumerate}
\item Use the ancient Egyptian method of doubling to calculate the product $13 \times 62.$ (See pages 51-52 for examples.)
\vspace{1.5in}
\item \textbf{IF} you were going to calculate the product $127 \times 1304$, how many times would you need to double 1304?\\

6 times because $2^7=128 > 127.$\\

\item Write the fraction $\large{\frac{7}{12}}$ as the sum of \emph{distinct unit fractions} in two different ways.\\

$\frac{1}{2}+\frac{1}{12}$ and $\frac{1}{3}+\frac{1}{4}$ 
\vfill
\end{enumerate}

\end{document}