% !TEX TS-program = pdflatexmk
\documentclass[12pt]{article}

% Layout.
\usepackage[top=1in, bottom=0.75in, left=1in, right=1in, headheight=1in, headsep=6pt]{geometry}

% Fonts.
\usepackage{mathptmx}
\usepackage[scaled=0.86]{helvet}
\renewcommand{\emph}[1]{\textsf{\textbf{#1}}}

% Misc packages.
\usepackage{amsmath,amssymb,latexsym}
\usepackage{graphicx,hyperref}
\usepackage{array}
\usepackage{xcolor}
\usepackage{multicol}
\usepackage{tabularx,colortbl}
\usepackage{enumitem}

\usepackage{fancyhdr}
\pagestyle{fancy} 
\lhead{\large\sf\textbf{MATH 316: History of Math}}
\rhead{\large\sf\textbf{Ticket 9 Sections 4.1-4.2}}

\begin{document}


\begin{enumerate}
\item Read Section 4.1. Summarize in 1-2 sentences.
\vfill
\item Read Section 4.2. (lightly). Summarize in 1-2 sentences.
\vfill
\item Review the 5 postulates on page 146. Explain what is peculiar about the 5th one. Draw it for yourself.
\vfill
\item For each common notion, write its algebraic equivalent.
\vfill
\item Read the statement and proof of Proposition 16 on page 152. What Common Notions, Postulates, and Propositions does the proof use?
\vfill 
\end{enumerate}
\end{document}