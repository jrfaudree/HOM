% !TEX TS-program = pdflatexmk
\documentclass[12pt]{article}

% Layout.
\usepackage[top=1in, bottom=0.75in, left=1in, right=1in, headheight=1in, headsep=6pt]{geometry}

% Fonts.
\usepackage{mathptmx}
\usepackage[scaled=0.86]{helvet}
\renewcommand{\emph}[1]{\textsf{\textbf{#1}}}

% Misc packages.
\usepackage{amsmath,amssymb,latexsym}
\usepackage{graphicx,hyperref}
\usepackage{array}
\usepackage{xcolor}
\usepackage{multicol}
\usepackage{tabularx,colortbl}
\usepackage{enumitem}

\usepackage{fancyhdr}
\pagestyle{fancy} 
\lhead{\large\sf\textbf{MATH 316: History of Math}}
\rhead{\large\sf\textbf{Ticket 20: Section 8.2}}

\begin{document}
\begin{enumerate}
\item Read Section 8.2 and answer the questions below.
\item Give three biographical details about Rene Descartes.
\vfill
\item \textbf{[A 3-line Problem]} Suppose you have three lines: $L_1: \: y=0$, $L_2: \: x=0,$ and $L_3: \: x=2.$ Describe the locus of points such that $d_1^2=3d_2d_3$ where $d_i$ is the distance to line $L_i.$  Your description should be both algebraic and graphical. The use of graphing tools is acceptable! 
\vfill
\item Give a 2-sentence summary of Descartes' celestial vortices theory. (pg 375-6)
\vfill
\end{enumerate}
\end{document}