% !TEX TS-program = pdflatexmk
\documentclass[12pt]{article}

% Layout.
\usepackage[top=1in, bottom=0.75in, left=1in, right=1in, headheight=1in, headsep=6pt]{geometry}

% Fonts.
\usepackage{mathptmx}
\usepackage[scaled=0.86]{helvet}
\renewcommand{\emph}[1]{\textsf{\textbf{#1}}}

% Misc packages.
\usepackage{amsmath,amssymb,latexsym}
\usepackage{graphicx,hyperref}
\usepackage{array}
\usepackage{xcolor}
\usepackage{multicol}
\usepackage{tabularx,colortbl}
\usepackage{enumitem}

\usepackage{fancyhdr}
\pagestyle{fancy} 
\lhead{\large\sf\textbf{MATH 316: History of Math}}
\rhead{\large\sf\textbf{Ticket 21: Section 8.2-8.3}}

\begin{document}
\begin{enumerate}
\item Read Section 8.2 and 8.3.
\item Give three biographical details about Isaac Newton.  State his dates and publications do not count.\\

Just FYI 1642-1727 (84 yrs)\\
As a young boy at public school, he was described as idle and inattentive and did not give a hint of his future accomplishments.\\
When he attended Cambridge, he had that analog of a work-study job and he started with the intent of studying law. Isaac Barrow, the first person to be the Lucasian Chair may have helped inspire his interest in mathematics and optics.\\
In 1696 (10 years after the publication of Principia), he was made warden and eventually master of the British mint. While he did not need to, he took the job seriously. He was buried in Westminster Abbey, should to shoulder with kings.\\
\vfill
\item Give three biographical details about Gottfried Leibniz.  Giving his dates and publications do not count. \\
FYI 1646-1716, (70 yrs)\\
Leipzig Germany\\
Started college at 15, the same year Newton did. He also pursued law.\\
He probably mostly learned mathematics from Christiaan Huygens a dutch mathematician living in Paris.\\
In his travels to England, he met Oldenburg, Pell, and Hooke.

\vfill
\end{enumerate}
\end{document}