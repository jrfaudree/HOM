% !TEX TS-program = pdflatexmk
\documentclass[12pt]{article}

% Layout.
\usepackage[top=1in, bottom=0.75in, left=1in, right=1in, headheight=1in, headsep=6pt]{geometry}

% Fonts.
\usepackage{mathptmx}
\usepackage[scaled=0.86]{helvet}
\renewcommand{\emph}[1]{\textsf{\textbf{#1}}}

% Misc packages.
\usepackage{amsmath,amssymb,latexsym}
\usepackage{graphicx,hyperref}
\usepackage{array}
\usepackage{xcolor}
\usepackage{multicol}
\usepackage{tabularx,colortbl}
\usepackage{enumitem}

\usepackage{fancyhdr}
\pagestyle{fancy} 
\lhead{\large\sf\textbf{MATH 316: History of Math}}
\rhead{\large\sf\textbf{Ticket 6 (\S 3.1-3.2)}}

\begin{document}


\begin{enumerate}
\item Read Burton \S 3.1. Summarize the \textbf{historical} topics discussed in these sections using at most two sentences.
\vspace{1in}
\item Read Burton \S 3.2. Summarize the \textbf{historical} topics discussed in these sections using at most two sentences.
\vspace{1in}
\item Provide biographical details about Thales of Miletus
	\begin{enumerate}
	\item time:
	\item location:
	\item list three mathematical accomplishments attributed to Thales
	\end{enumerate}
\vfill
\item Provide biographical details about Pythagoras of Samos
	\begin{enumerate}
	\item time:
	\item location:
	\item list three mathematical accomplishments attributed to Pythagoras
	\end{enumerate}
\vfill
\newpage
\item A tall spruce tree stands in the yard of a house. 
	\begin{enumerate}
	\item Describe a strategy for determining the height of the tree analagous to that Thales used to determine the height of the Great Pyramid. Pick your own sample numbers to illustrate.
	\vfill
	\item What assumption are \textbf{implied} in the algorithm used in part (a).
	\vfill
	\end{enumerate}
\item The symbols $t_n$ and $s_n$ are introduced on page 95.
	\begin{enumerate}
	\item What do these symbols represent?
	\vfill
	\item It is true that $t_n=1+2+\cdots+n$ and that $1+2+\cdots n=\frac{(n+1)n}{2}.$
		\begin{enumerate}
		\item What technique would a modern mathematics student use show the second equality above? (Indeed you probably proved this in Introduction to Proofs or Discrete Math.)
		\vfill
		\item What is a reasonable guess about how the Pythagoreans deduced that equality?
		\vfill
		\end{enumerate}
	\end{enumerate}
\end{enumerate}
\end{document}