% !TEX TS-program = pdflatexmk
\documentclass[12pt]{article}

% Layout.
\usepackage[top=1in, bottom=0.75in, left=1in, right=1in, headheight=1in, headsep=6pt]{geometry}

% Fonts.
\usepackage{mathptmx}
\usepackage[scaled=0.86]{helvet}
\renewcommand{\emph}[1]{\textsf{\textbf{#1}}}

% Misc packages.
\usepackage{amsmath,amssymb,latexsym}
\usepackage{graphicx,hyperref}
\usepackage{array}
\usepackage{xcolor}
\usepackage{multicol}
\usepackage{tabularx,colortbl}
\usepackage{enumitem}

\usepackage{fancyhdr}
\pagestyle{fancy} 
\lhead{\large\sf\textbf{MATH 316: History of Math}}
\rhead{\large\sf\textbf{Ticket 19: Section 8.2}}

\begin{document}

Jill's Solutions

\begin{enumerate}
\item Read Section 8.2 and answer the questions below.
\item Describe Galileo Galilei's \textit{Dialogue Concerning the Two Chief World Systems}. (Hereafter called \textit{Dialogue}. When was it published? What was it about? What was its impact?\\

Published in 1632. It was a dialogue between three people about whether the Earth rotated about sun (helio-centric view) or the sun went around the Earth (geo-centric). One person argued in favor of the helio-centric view, on in favor of the geo-centric, and a third moderator. Since Galileo knew the helio-centric theory to be correct, the character arguing this view won the day.

It was written in Italian (not Latin) and thus accessible to literate people beyond academics and had no math. Consequently it was read by many (selling out as it came off the presses). Galileo was jailed for this publication.
\vfill
\item Return to the bottom of page 345 - 346. List three ``firsts" in modern algebraic notation. State the notation, the name of the inventor, the text in which it appeared and the date. For example, in 1557, Robert Recorde published \textit{The Whetstone of Witte} introducing the symbol $=$ (thought it was somewhat longer!).\\

So many:\\
\begin{itemize}
\item 1552, Chrostoff Rudolff \textit{Die Coss}, the radical for a root (though missing the upper bar)
\item 1659, Rahn \textit{Teutshe Algebra} the use of $\div$ for division.
\item1630, \textit{Key to Mathematics}, Willia Oughtred, the use of $\times$ for multiplication .
\end{itemize}
\vfill
\item Return to the bottom of page 347 and describe the advance in algebraic notation attributed to Francois Vieta and Rene Descartes. Include approximate dates.

They distinguished between variables and constants.

Francois Vieta (1540-1603) Publication around 1891, Rene Descartes (1596-1650) Publication around 1637.

\vfill
\item What was the substance of Simon Stevins \textit{The Tenth} and approximately when was it published?

How to write using decimal notation -- that is fractions base 10 as opposed to integers base 10 and fraction base 60. (Note the we still go back and forth on this.) In particular, he observed that using decimal notation means integer-based algorithms work for nonintegers.
\vfill
\item Provided approximate dates for Nicolas Copernicus, Johannes Kepler, and Tycho Brache and a short summary of their contributions to astronomy.

Eratosthenes (230 BC) Calculated the radius of the Earth (ie it's round)
Ptolemy (85-160) Geo-centric. Circular orbits.
Copernicus (1473-1543) Helio-centric. Circular orbits and epicycles.
Brache (1546-1601) Data collection.
Kepler (1571-1630) Helio-centric. Elliptical orbits.
\vfill
\end{enumerate}
\end{document}