% !TEX TS-program = pdflatexmk
\documentclass[12pt]{article}

% Layout.
\usepackage[top=1in, bottom=0.75in, left=1in, right=1in, headheight=1in, headsep=6pt]{geometry}

% Fonts.
\usepackage{mathptmx}
\usepackage[scaled=0.86]{helvet}
\renewcommand{\emph}[1]{\textsf{\textbf{#1}}}

% Misc packages.
\usepackage{amsmath,amssymb,latexsym}
\usepackage{graphicx,hyperref}
\usepackage{array}
\usepackage{xcolor}
\usepackage{multicol}
\usepackage{tabularx,colortbl}
\usepackage{enumitem}

\usepackage{fancyhdr}
\pagestyle{fancy} 
\lhead{\large\sf\textbf{MATH 316: History of Math}}
\rhead{\large\sf\textbf{Ticket 6 (\S 3.1-3.2)}}

\begin{document}

\begin{center} Jill's Solutions \end{center}
\begin{enumerate}
\item Read Burton \S 3.1. Summarize the \textbf{historical} topics discussed in these sections using at most two sentences.\\

This section describes the evolution and expansion of the Greek culture throughout the Mediterranean during roughly the period from 800-600 BC. It emphasizes that the spread-out geography and improvements in writing language contributed to a political and intellectual system in which power and knowledge were more diffuse and less concentrated when compared to say earlier Egyptian ones.\\
\item Read Burton \S 3.2. Summarize the \textbf{historical} topics discussed in these sections using at most two sentences.\\

This sections mostly discusses the nature of the cult of Pythagoras and its substantial and long-lasting influence of mathematics and eduction, including the rough framework for what we call today a liberal arts education.\\

\item Provide biographical details about Thales of Miletus
	\begin{enumerate}
	\item time: 622-547 BC
	\item location: Miletus in Ionia (now Turkey)
	\item list three mathematical accomplishments attributed to Thales. (Many choices here!)\\
	\begin{itemize}
	\item Introduction of logical formal proofs
	\item Many geometric facts such as the ratio of sides of similar triangles are equal and the opposite angles of intersecting straight lines are equal
	\item Using similar triangles to solve practical problems such as the height of objects or distance of ships at sea.
	\end{itemize}
	\end{enumerate}

\item Provide biographical details about Pythagoras of Samos
	\begin{enumerate}
	\item time: 585-500 BC
	\item location: Born in Samos in Ionia. Travelled around the Mediterranean. Set up up school in Croton in present day Italy.
	\item list three mathematical accomplishments attributed to Pythagoras
		\begin{itemize}
		\item Some elementary number theory
		\item The square root of 2 is irrational
		\item The Pythagorean Theorem
		\end{itemize}
	\end{enumerate}

\item A tall spruce tree stands in the yard of a house. 
	\begin{enumerate}
	\item Describe a strategy for determining the height of the tree analagous to that Thales used to determine the height of the Great Pyramid. Pick your own sample numbers to illustrate.\\
	
	On a sunny day, have my 6 foot tall husband stand and I measure his shadow to be 5 feet long. At that same time, I measure the shadow of the tree to be 35 feet long. Using similar triangles and $h$ to represent the height of the tree, we know $\frac{h}{35}=\frac{6}{5}$ or $h=42$ feet.
	
	\item What assumption are \textbf{implied} in the algorithm used in part (a).\\
	That the ground is level and the tree is growing orthogonal to the ground.\\
	
	\end{enumerate}
\item The symbols $t_n$ and $s_n$ are introduced on page 95.
	\begin{enumerate}
	\item What do these symbols represent?\\
	
	Specific figurative numbers: triangular and square numbers respectively.\\

	\item It is true that $t_n=1+2+\cdots+n$ and that $1+2+\cdots n=\frac{(n+1)n}{2}.$
		\begin{enumerate}
		\item What technique would a modern mathematics student use show the second equality above? (Indeed you probably proved this in Introduction to Proofs or Discrete Math.)\\
		
		Mathematical induction.
		
		\item What is a reasonable guess about how the Pythagoreans deduced that equality?\\
		
		Pictures made of dots rearranged in various ways.
		\end{enumerate}
	\end{enumerate}
\end{enumerate}
\end{document}