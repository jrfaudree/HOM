% !TEX TS-program = pdflatexmk
\documentclass[12pt]{article}

% Layout.
\usepackage[top=1in, bottom=0.75in, left=1in, right=1in, headheight=1in, headsep=6pt]{geometry}

% Fonts.
\usepackage{mathptmx}
\usepackage[scaled=0.86]{helvet}
\renewcommand{\emph}[1]{\textsf{\textbf{#1}}}

% Misc packages.
\usepackage{amsmath,amssymb,latexsym}
\usepackage{graphicx,hyperref}
\usepackage{array}
\usepackage{xcolor}
\usepackage{multicol}
\usepackage{tabularx,colortbl}
\usepackage{enumitem}

\usepackage{fancyhdr}
\pagestyle{fancy} 
\lhead{\large\sf\textbf{MATH 316: History of Math}}
\rhead{\large\sf\textbf{Ticket 1 (\S 1.1-1.2)}}

\begin{document}


\begin{enumerate}
\item Read Burton \S 1.1-1.2. Summarize the mathematical topic discussed in these sections using at most two sentences.
\vspace{1in}
\item For each item/numerical system mentioned in this section, state the rough time period, an approximate location, and a brief description of the numerical system.
	\begin{enumerate}
	\item Ishango Bone (pictured on page   5)
		\begin{enumerate}
		\item time period:
		\item location:
		\item description:
		\vfill
		\end{enumerate}
	\item quipus
		\begin{enumerate}
		\item time period:
		\item location:
		\item description:
		\vfill
		\end{enumerate}
	\item Mayan symbols
		\begin{enumerate}
		\item time period:
		\item location:
		\item description:
		\vfill
		\end{enumerate}
	\item Heiroglyphs (pages 13-14)
		\begin{enumerate}
		\item time period:
		\item location:
		\item description:
		\vfill
		\end{enumerate}
	\item Hieratic Symbols (pages 15-16)
		\begin{enumerate}
		\item time period:
		\item location:
		\item description:
		\vfill
		\end{enumerate}
	\item Alphabetic System (page 16-17)
		\begin{enumerate}
		\item time period:
		\item location:
		\item description:
		\vfill
		\end{enumerate}
	\end{enumerate}
\end{enumerate}

\end{document}