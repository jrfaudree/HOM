% !TEX TS-program = pdflatexmk
\documentclass[12pt]{article}

% Layout.
\usepackage[top=1in, bottom=0.75in, left=1in, right=1in, headheight=1in, headsep=6pt]{geometry}

% Fonts.
\usepackage{mathptmx}
\usepackage[scaled=0.86]{helvet}
\renewcommand{\emph}[1]{\textsf{\textbf{#1}}}

% Misc packages.
\usepackage{amsmath,amssymb,latexsym}
\usepackage{graphicx,hyperref}
\usepackage{array}
\usepackage{xcolor}
\usepackage{multicol}
\usepackage{tabularx,colortbl}
\usepackage{enumitem}

\usepackage{fancyhdr}
\pagestyle{fancy} 
\lhead{\large\sf\textbf{MATH 316: History of Math}}
\rhead{\large\sf\textbf{Homework 10}}

\begin{document}
\begin{enumerate}
\item Probability: Problem of Points\\

The game\\
A fair coin is repeatedly flipped. If the coin lands as a heads (H), Hilda gets a point. If the coin lands tails, Tom gets a point. The first person to 5 points wins. Each player pays \$10 to play.

You will fill out the table below indicating how much each player should get assuming the game is stopped before finishing. You must justify your calculations using \textbf{both} Pascal's method and Fermat's method.\\

\bgroup
\def\arraystretch{1.5}
\begin{tabular}{l || c || c|c}
& status of game&payout to& payout to\\
&at termination& Hilda&Tom\\
\hline \hline
a)& Hilda 4, Tom 4 &&\\
\hline
b)& Hilda 4, Tom 3 &&\\
\hline
c)& Hilda 4, Tom 2 &&\\
\hline
d)& Hilda 4, Tom 1 &&\\
\hline
e)& Hilda 4, Tom 0 &&\\
\hline
f)& Hilda 3, Tom 2 &&\\
\hline
\end{tabular}
\egroup

\item In class, I described Pascal's Method as \emph{recurrsive}. Using your work from problem 1, explain, in your own words, why this is an accurate description of this method.
\item Give an argument that Fermat's strategy is clear and correct theoretically but may be challenging to implement in specific instances.
\item (perfect numbers) 
	\begin{enumerate}
	\item Use the definition to show that 496 is a perfect number.
	\item Show that $496=2^{k-1}(2^k-1)$ for some integer $k$ and $2^k-1$ is prime.
	\end{enumerate}
\item Euler and Number Theory\\
Recall that Euclid proved: If $2^k-1$ is prime ($k>1$), then $n=2^{k-1}(2^k-1)$ is perfect.\\
Euler proved that if $n$ is an even perfect number, then $n=2^{k-1}(2^k-1)$ for some integer $k$ and $2^k-1$ is prime. \\
Complete the following using the results above.
	\begin{enumerate}
	\item Show that $8128=2^{6}(2^{7}-1)$ is perfect.
	\item Show that $2096128=2^{10}(2^{11}-1)$ is not perfect.
	\item We know there are an infinite number of primes. Are there an infinite number of perfect numbers?
	\item What is the first odd perfect number?
	\end{enumerate}
\item Use Euler's method to show $\displaystyle{\sum_{i=1}^\infty \frac{1}{2i-1}= \frac{1}{1^2}+\frac{1}{3^2}+\frac{1}{5^2}+\frac{1}{7^2}+\cdots= \frac{\pi^2}{8}.}$

Hint: Use the Maclaurin Series for $\cos(x)$ and the fact that $\cos(x)=0$ has roots $\pm\frac{\pi}{2},$ $\pm\frac{3\pi}{2},$ $\pm\frac{5\pi}{2},$ ...\end{enumerate}
\end{document}